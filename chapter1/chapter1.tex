\chapter{Elliptic Curves}
\label{chapter:ellibase}

The origin of the elliptic curves goes back to the 18th century when mathematicians tried to 
calculate the arc length of an ellipse. This led to the study of elliptic integrals, and then 
elliptic functions, and the brilliant works of Weierstrass. Of course, all these were happening in 
the complex field $\vmathbb{C}$. But the theory was extended over arbitrary fields, specially finite 
fields, afterwards, and today, there is a huge literature on elliptic curves and their applications. 
In this chapter, we give a brief introduction to the basic concepts of the theory of elliptic curves 
over general fields. 







\section{The Weierstrass equation and the group law}
\label{section:W-grouplaw}

Let $k$ be a field. The set $\vmathbb{A}_{k}^2 = \{ (x, y) \in k \times k \}$ is called the affine 
plane over $k$. Any nonconstant squarefree polynomial $f \in k[x,y]$ defines an affine plane curve 
$C_f$ whose points are the zero set of $f$ in $\vmathbb{A}_k^2$. We say that $C_f$ is defined over 
$k$. For an extension $L \supseteq k$, the zero set of $f$ in $\vmathbb{A}_L^2$ is denoted by 
$C_f(L)$. The projective plane over $k$, denoted by $\vmathbb{P}_k^2$, is the set of all triples $(x, 
y, z) \in k^3$ with $(x, y, z) \ne (0, 0, 0)$ modulo an equivalence relation $\sim$ where $(x, y, z) 
\sim (x_1, y_1, z_1)$ if and only if there exists a nonzero $\mu \in k$ such that $(z_1, y_1, z_1) = 
(\mu x, \mu y, \mu z)$. The class of the point $(x, y, z)$ is denoted by $(x: y: z)$. A projective 
plane curve is defined similar to the affine curve except that the defining polynomial $f \in k[x, 
y, z]$ should be homogeneous, otherwise the zero set of $f$ in $\vmathbb{P}_k^2$ is not well defined. 
A point $(x: y: z) \in \vmathbb{P}_k^2$ is a finite point if $z \ne 0$, and a point at infinity if $z 
= 0$. There is a natural embedding
$$
\setlength\arraycolsep{2pt}
\begin{array}{llll}
\varphi: & \vmathbb{A}_k^2 & \hookrightarrow & \vmathbb{P}_k^2 \\
 & (x, y) & \mapsto & (x: y: 1)
\end{array}
$$
A plane curve $C_f$ is said to be singular at a point $P$ if $\frac{\partial f}{\partial x}(P) = 
\frac{\partial f}{\partial y}(P) = 0$, where the partial derivative of a polynomial is defined in 
the usual way, and nonsingular at $P$ otherwise. The curve $C_f$ is nonsingular if it has no 
singular points. An \emph{elliptic curve} over $k$, denoted by $E_k$, is a nonsingular projective 
plane curve defined over $k$ by a polynomial of the form $f(x, y) = y^2z + a_1xyz + a_3yz^2 - (x^3 + 
a_2x^2z + a_4xz^2 + a_6z^3) \in k[x, y, z]$. Thus, the points of $E_k$ are the solution set in 
$\vmathbb{P}_k^2$ of an equation of the form
\begin{equation}
\label{equation:PEC}
E_k: \quad y^2z + a_1xyz + a_3yz^2 = x^3 + a_2x^2z + a_4xz^2 + a_6z^3
\end{equation}
with $a_1, a_3, a_2, a_4, a_6 \in k$. If we let $z = 0$ in \refequation{equation:PEC} then $x^3 = 
0$, and hence $(0: 1: 0)$ is the only point at infinity of $E_k$. Therefore, all point of $E_k$ are 
of the form $(x: y: 1)$, i.e. are in the finite plane, except the above point which we denote by 
$\infty$. So, the set of points on $E_k$ is the solution set in $\vmathbb{A}_k^2$ of 
\begin{equation}
\label{equation:AEC}
\quad y^2 + a_1xy + a_3y = x^3 + a_2x^2 + a_4x + a_6
\end{equation}
together with $\infty$.  For an extension $L \supseteq k$, the points on $E_k$ with coordinates in 
$L$ will be denoted by $E_k(L)$, i.e. 
$$
E_k(L) = \{ \infty \} \cup \{ (x, y) \in \vmathbb{A}_L^2 \mid y^2 + a_1xy + a_3y = x^3 + a_2x^2 + 
a_4x + a_6\}
$$
Equation (\ref{equation:AEC}) is called the generalized Weierstrass equation of $E_k$. If the 
characteristic of $k$ is not $2$ then, applying the change variables $y \mapsto y - a_1x / 2 - a_3 / 
2$, \refequation{equation:AEC} can be rewritten as
\begin{equation}
\label{equation:AEC2}
\quad y^2 = x^3 + \frac{b_2}{4}x^2 + \frac{b_4}{2}x + \frac{b_6}{4}
\end{equation}
where  $b_2 = a_1^2 + 4a_2,\: b_4 = a_1a_3 + 2a_4,\: b_6 = a_3^2 + 4a_6$. Let us also define $b_8 = 
a_1^2a_6 + 4a_2a_6 - a_1a_3a_4 + a_2a_3^2 - a_4^2$ for future references. If the characteristic is 
also not $3$ then the change of variables $x \mapsto x - b_2 / 7$ results in
\begin{equation}
\label{equation:AEC3}
\quad y^2 = x^3 + Ax + B
\end{equation}
for some constants $A, B \in k$. Equation (\ref{equation:AEC3}) is called the Weierstrass equation 
of $E_k$. We shall simply use $E$ instead of $E_k$ when $k$ is uniquely known from the context. 
Unless otherwise specified, by an elliptic curve $E$ we shall mean an elliptic curve defined by 
\refequation{equation:AEC3}. Since $E$ is nonsingular, the cubic $x^3 + Ax + B$ cannot have repeated 
roots and hence $\Delta = -4A^3 - 27B^2 \ne 0$ where $\Delta$ is the discriminant of the cubic. So, 
we always have $4A^3 + 27B^2 \ne 0$. Let $P = (x_P, y_P)$ and $Q = (x_Q, y_Q)$ be two points on $E$. 
Define the addition $R = (x_R, y_R) = P + Q$ as follows. For any point $P$, $P + \infty = P$, so for 
example $\infty + \infty = \infty$. For $P, Q \ne \infty$:
\begin{enumerate}
\item If $x_P \ne x_Q$ then $x_R = \left( \frac{y_Q - y_P}{x_Q - x_P} \right)^2 - x_P - x_Q$ and 
$y_R = \frac{y_Q - y_P}{x_Q - x_P}(x_P - x_R) - y_P$.
\label{item:glaw-dist}
\item If $x_P = x_Q$ but $y_P \ne y_Q$ then $P + Q = \infty$.
\item If $P = Q$ and $y_P \ne 0$ then $x_R = \left( \frac{3x_P^2 + A}{2y_P} \right)^2 - 2x_P$ and 
$y_R = \frac{3x_P^2 + A}{2y_P}(x_P - x_R) - y_P$.
\item If $P = Q$ and $y_P = 0$ then $P + Q = \infty$.
\end{enumerate}
It is not hard to prove that under the above addition, the points on $E$ form an abelian group with 
$\infty$ as identity (e.g. \cite[Sec. 2.4]{Washington2008} or \cite[Sec. 2.11]{Enge1999}). From rule 
2 and 4, and \refequation{equation:AEC3} we have $-P = (x_P, -y_P)$. Figure \ref{figure:elli-glaw} 
illustrates the geometrical view of the group law for the curve $y^2 = x^3 + x^2 - 2x$ over the real 
field $\vmathbb{R}$ when $x_P \ne x_Q$, i.e. Case \ref{item:glaw-dist} of the above. The two ends of 
the $y$-axis are labelled with $\infty$ to show that they meet at infinity.

\begin{figure}[ht]
\setlength{\abovecaptionskip}{-0.5cm}
%\setlength{\belowcaptionskip}{0cm}
\begin{center}
\begin{tikzpicture}[scale=0.85]
%\draw[very thin, color=gray] (-3.9, -3.9) grid (5.9, 3.9);
\draw[->] (-4, 0)--(6, 0) node[right] {$x$};
\draw[->] (0, -4) node[below] {$\infty$} --(0, 4) node[right] {$y$} node[above] {$\infty$};
\draw[color=blue, samples=100, smooth, domain=-2:0] plot (\x, {0.75 * sqrt(\x*(\x + 2)*(\x - 1))});
\draw[color=blue, samples=100, smooth, domain=-2:0] plot (\x, {-0.75 * sqrt(\x*(\x + 2)*(\x - 1))});
\draw[color=blue, samples=100, smooth, domain=1:3] plot (\x, {0.75 * sqrt(\x*(\x + 2)*(\x - 1))});
\draw[color=blue, samples=100, smooth, domain=1:3] plot (\x, {-0.75 * sqrt(\x*(\x + 2)*(\x - 1))}) 
node[right] {$y^2 = x^3 + x^2 - 2x$};
\draw[color=red, samples=100, smooth, domain=-3:4] plot (\x, {0.49*(\x+0.7)+0.93});
%----------------------------------------
\coordinate[label=above:$P$] (P) at (-0.7, 0.93);
\coordinate[label=above left:$Q$] (Q) at (-1.97, 0.31);
\coordinate[label=above left:$R'$] (R) at (2.1, 2.31);
\coordinate[label=left:$R$] (S) at (2.1, -2.31);
%----------------------------------------
\fill [red] (P) circle (1.3pt);
\fill [red] (Q) circle (1.3pt);
\fill [red] (R) circle (1.3pt);
\fill [red] (S) circle (1.3pt);
%----------------------------------------
\draw[color=orange, densely dashed] (2.1, 2.25)--(2.1, -2.25);  
\end{tikzpicture}
\end{center}
\caption{$P + Q = R$}
\label{figure:elli-glaw}
\end{figure}







\section{Endomorphisms}
Let $E^1$ and $E^2$ be elliptic curves defined over the field $k$ and let $\overline{k}$ be the 
algebraic closure of $k$. An \textbf{isogeny} from $E^1$ to $E^2$ is a group homomorphism
$$
\setlength\arraycolsep{2pt}
\begin{array}{llll}
\sigma: & E^1(\overline{k}) & \longrightarrow & E^2(\overline{k}) \\
& (x, y) & \longmapsto & (f(x, y), g(x, y))
\end{array}
$$
where $f, g \in \overline{k}(x, y)$ are rational functions. Replacing $y^2$ by $x^3 + Ax + B$ in 
$f(x, y)$ we can write $f(x, y) = (a(x) + yb(x)) / (c(x) + yd(x))$ where $a, b, c, d$ are 
polynomials. Removing $y$ from the denominator by multiplying $c(x) - yd(x)$ on both denominator and 
numerator, and using the fact that $(f(x, -y), g(x, -y)) = \sigma(x, -y) = \sigma(-(x, y)) = (f(x, 
y), -g(x, y))$ we have $f(x, y) = r(x)$ where $r(x)$ is a rational function in $x$. By a similar 
process $g(x, y) = ys(x)$ for some rational function $s(x)$. So, we can always write $\sigma(x, y) = 
(r(x), ys(x))$ for some rational functions $r$ and $s$.

Let $r(x) = r_1(x) / r_2(x)$ where $\gcd(r_1(x), r_2(x)) = 1$. The degree of $\sigma$ is defined to 
be $\max\{ \deg r_1(x), \deg r_2(x)\}$ when $\sigma$ is nontrivial. If $\sigma = 0$ then $\deg 
\sigma = 0$. A nontrivial isogeny $\sigma$ is \textbf{separable} if the derivative $r'(x)$ is not 
the zero function.
\begin{lemma}
Every nontrivial isogeny $\sigma: E^1(\overline{k}) \rightarrow E^2(\overline{k})$ of elliptic 
curves is surjective.
\end{lemma}
\begin{proof}
See \cite{Washington2008} for an elementary proof, and \cite{Fulton2008} for a proof for general 
projective curves.
\end{proof}
By the kernel of an isogeny $\sigma$ we mean its kernel as a group homomorphism.
\begin{proposition}
\label{proposition:iso-ker}
For any nontrivial isogeny $\sigma$, $\#\ker(\sigma) \le \deg(\sigma)$ with equality if $\sigma$ is 
separable. 
\end{proposition}
\begin{proof}
See \cite[Sec. 3.4]{Silverman2009} or \cite{Washington2008}.
\end{proof}
An \textbf{endomorphism} of elliptic curve $E$ is an isogeny $\varphi: E(\overline{k}) \rightarrow 
E(\overline{k})$. The set of all endomorphisms of $E$ is denoted by $\ringofend(E)$. For every two 
endomorphisms $\alpha, \beta \in \ringofend(E)$ define $(\alpha + \beta)(P) = \alpha(P) + \beta(P)$, 
and $(\alpha\beta)(P) = \alpha(\beta(P))$ for $P \in E$. Then it can easily be verified that 
$\ringofend(E)$ is a ring. Two basic maps on $E$ are
$$
\setlength\arraycolsep{2pt}
\begin{array}{llll}
\text{id}: & E(\overline{k}) & \rightarrow & E(\overline{k}) \\
& P & \mapsto & P
\end{array}
\qquad
\text{ and }
\qquad
\begin{array}{llll}
[n]: & E(\overline{k}) & \rightarrow & E(\overline{k}) \\
& P & \mapsto & [n]P = P + P + \cdots + P \quad (n \text{ times})
\end{array}
$$
The first one is clearly an endomorphism. Using the addition laws and a simple induction shows that 
the map $[n]$ is also an endomorphism. This means that $\ringofend(E)$ always contains a copy of the 
integer ring $\vmathbb{Z}$. We will show that $[n]$ is separable if and only if $n$ is relatively 
prime to $\fieldchar(k)$. To this end, let first prove the following result.
\begin{lemma}
\label{lemma:mult-by-n-diff}
Let $\varphi_1, \varphi_2 \in \ringofend(E)$ such that $\varphi_i(x, y) = (r_i(x), ys_i(x))$, $i = 
1, 2$ where $r_i, s_i$ are rational functions. Let $\varphi(x, y) = (r(x), ys(x))$ such that 
$\varphi = \varphi_1 + \varphi_2$. If $r_i'(x) / s_i(x) = c_i$ for some constants $c_i, \: i = 1, 2$ 
then $r'(x) / s(x) = c_1 + c_2$.
\end{lemma}
\begin{proof}
Let $\tau_Q(x, y) = (f, g)$ be the translation-by-$Q$ map on $E$. Then it is straightforward, but 
lengthy, to show that $\frac{\partial f}{\partial x} + \frac{dy}{dx}\frac{\partial f}{\partial y} = 
\frac{g}{y}$. Let $\varphi(x, y) = (x_3, y_3)$, and let $\varphi_i(x, y) = (x_i, y_i)$, $i = 1, 2$. 
If we let $Q = (x_1, y_1)$ then $\frac{\partial x_3}{\partial x_2} + \frac{dy_2}{dx_2}\frac{\partial 
x_3}{\partial y_2} = \frac{y_3}{y_2}$, and if $Q = (x_2, y_2)$ then $\frac{\partial x_3}{\partial 
x_1} + \frac{dy_1}{dx_1}\frac{\partial x_3}{\partial y_1} = \frac{y_3}{y_1}$. Therefore
\begin{align*}
r'(x) 
&= \frac{dx_3}{dx} = \frac{dx_1}{dx}\frac{\partial x_3}{\partial x_1} + 
\frac{dx_2}{dx}\frac{\partial x_3}{\partial x_2} + \frac{dx_1}{dx}\frac{dy_1}{dx_1}\frac{\partial 
x_3}{\partial y_1} + \frac{dx_2}{dx}\frac{dy_2}{dx_2}\frac{\partial x_3}{\partial y_2} \\
&= \frac{dx_1}{dx}\frac{y_3}{y_1} + \frac{dx_2}{dx}\frac{y_3}{y_2} = c_1\frac{y_3}{y} + 
c_2\frac{y_3}{y} = (c_1 + c_2)s(x)
\end{align*}
as desired.
\end{proof}
\begin{corollary}
\label{corollary:mult-by-n}
Let $P = (x, y)$ be a point on $E$, and let $[n]P = (r(x), ys(x))$ for some rational functions $r, 
s$. Then $r'(x) / s(x) = n$. Therefore, the mapping $[n]$ is separable if and only if $\gcd(n, 
\fieldchar(k)) = 1$.
\end{corollary}
\begin{proof}
The statement is clear for $n = 1$. Assume it is true for all $k < n$. Let $[n - 1]P = (r_1(x), 
ys_1(x))$. Then $r_1'(x) / s_1(x) = n - 1$. We have $(r(x), ys(x)) = [n]P = [n - 1]P + P = (r_1(x), 
ys_1(x)) + (x, y)$, so by \reflemma{lemma:mult-by-n-diff}, $r'(x) / s(x) = n - 1 + 1 = n$.
\end{proof}
We will give explicit formulae for the endomorphism $[n]$ in \refsection{section:div-poly}. In the 
theory of elliptic curves over finite fields, i.e. $k = \vmathbb{F}_q$, the following map plays an 
important role.
\begin{equation}
\label{equation:Frobenius}
\setlength\arraycolsep{2pt}
\begin{array}{llll}
\phi_q: & E(\overline{\vmathbb{F}_q}) & \rightarrow & E(\overline{\vmathbb{F}_q}) \\
& (x, y) & \mapsto & (x^q, y^q), \: \infty \mapsto \infty
\end{array}
\end{equation}
It is called the Frobenius map. It is, indeed, the curve version of the Frobenius automorphism 
$\phi_q: \overline{\vmathbb{F}_q} \rightarrow \overline{\vmathbb{F}_q}$. Since $a^q = a$ for every $a 
\in \vmathbb{F}_q$, the field $\vmathbb{F}_q$ is characterized by $\phi_q$. Thus, applying $\phi_q$ to 
the equation of $E$, it is clear that $\phi_q(x, y) \in E(\overline{\vmathbb{F}_q})$. We also have $P 
\in E(\vmathbb{F}_q)$ if and only if $\phi_q(P) = P$. By the group laws defined in 
\refsection{section:W-grouplaw}, it can be easily seen that $\phi$ is an endomorphism of $E$. Also 
it is clear that $\phi_q$ is not separable. In fact we have
\begin{corollary}
\label{corollary:Frob-sep}
Let $E$ be an elliptic curve defined over $\vmathbb{F}_q$. For integers $a$ and $b$, not both zero, 
and the Frobenius endomorphism $\phi_q$, the endomorphism $a\phi_q + b$ is separable if and only if 
$\gcd(b, q) = 1$.
\end{corollary}
\begin{proof}
\reflemma{lemma:mult-by-n-diff} and \refcorollary{corollary:mult-by-n}.
\end{proof}
\begin{proposition}
\label{proposition:Frob-ker-E}
Let $E$ be an elliptic curve defined over $\vmathbb{F}_q$, and $\phi_q$ the Frobenius endomorphism. 
Then 1. $\#E(\vmathbb{F}_{q^n}) = \deg(\phi_q^n - 1)$, and 2. $\ker(\phi_q^n - 1) = 
E(\vmathbb{F}_{q^n})$
\end{proposition}
\begin{proof}
The map $\phi_q^n$ is equivalent to the Frobenius map on $E(\overline{\vmathbb{F}_{q^n}})$. So, 
$(\phi_q^n - 1)(P) = 0$ if and only if $P \in E(\vmathbb{F}_{q^n})$ which proves 2. By 
\refcorollary{corollary:Frob-sep}, $(\phi_q^n - 1)$ is separable hence $\#E(\vmathbb{F}_{q^n}) = 
\#ker(\phi_q^n - 1) = \deg(\phi_q^n - 1)$ by \refproposition{proposition:iso-ker}. This proves 1.
\end{proof}









\section{Division polynomials}
\label{section:div-poly}

For a positive integer $n$ and a point $P$ on an elliptic curve $E$, $[n]P$ can be computed using 
the repeated squaring algorithm. Another way of computing $[n]P$ is using explicit formulae 
expressed by division polynomials. The division polynomials for an elliptic curve defined by 
\refequation{equation:AEC} are defined recursively as follows.
\begin{align*}
& \psi_0 = 0 \\
& \psi_1 = 1 \\
& \psi_2 = 2y + a_1x + a_3 \\
& \psi_3 = 3x^4 + b_2x^3 + 3b_4x^2 + 3b_6x + b_8 \\
& \psi_4 = \psi_2 \cdot (2x^6 + b_2x^5 + 5b_4x^4 + 10b_6x^3 + 10b_8x^2 + (b_2b_8 - b_4b_6)x + 
(b_4b_8 - b_6^2)) \\
& \psi_{2m + 1} = \psi_{m + 2}\psi_m^3 - \psi_{m - 1}\psi_{m + 1}^3 & \text{ for } m \ge 2\\
& \psi_{2m} = \psi_2^{-1}\psi_m(\psi_{m - 1}^2\psi_{m + 2} - \psi_{m - 2}\psi_{m + 1}^2) & \text{ 
for } m \ge 2
\end{align*}
where $b_2, b_4, b_6$, and $b_8$ are the values defined in \refequation{equation:AEC2}. Also define
\begin{align*}
\phi_m &= x\psi_m^2 - \psi_{m - 1}\psi_{m + 1} \\
\omega_m &= (2\psi_m)^{-1}(\psi_{2m} - (a_1\phi_m + a_3\psi_m^2)\psi_m^2)
\end{align*}
where $\omega_m$ is defined when the field $k$ is not of characteristic $2$. In the following let $R 
= \vmathbb{Z}[a_1, \dots, a_6, x, \psi_2^2]$.
\begin{lemma}
\label{lemma:divpoly-in-R}
If $m$ is odd then $\psi_m \in R$. If $m$ is even then $(\psi_2)^{-1}\psi_m \in R$.
\end{lemma}
\begin{proof}
The lemma is clear for $m \le 4$. Assume it is true for $\vmathbb{N}_{< k}$. If $k$ is odd, say $k = 
2\ell + 1$, then $\psi_k = \psi_{\ell + 2}\psi_\ell^3 - \psi_{\ell - 1}\psi_{\ell + 1}^3$. If $\ell$ 
is odd then $\psi_{\ell - 1}\psi_{\ell + 1}^3 \in \psi_2^4R \subseteq R$ hence $\psi_k \in R$. If 
$\ell$ is even then $\psi_{\ell + 2}\psi_\ell^3 \in \psi_2^4R \subseteq R$ hence $\psi_k \in R$. If 
$k$ is even, say $k = 2\ell$, then $\psi_k = \psi_2^{-1}\psi_\ell(\psi_{\ell - 1}^2\psi_{\ell + 2} - 
\psi_{\ell - 2}\psi_{\ell + 1}^2)$. If $\ell$ is odd then $\psi_{\ell - 1}^2, \psi_{\ell + 1}^2 \in 
\psi_2^2R$ hence $\psi_k \in \psi_2^{-1}(\psi_2^2R) = \psi_2R$. If $\ell$ is even then $\psi_\ell, 
\psi_{\ell + 2}, \psi_{\ell - 2} \in \psi_2R$ hence $\psi_k \in \psi_2^{-1}(\psi_2^2R) = \psi_2R$.
\end{proof}
A direct consequence of \reflemma{lemma:divpoly-in-R} is that $\psi_m^2, \phi_m \in R$ for all $m$. 
From \refequation{equation:AEC2} we have $\psi_2^2 = (2y + a_1x + a_3)^2 = 4(y^2 + a_1xy + a_3y) + 
(a_1x + a_3)^2 = 4(x^3 + a_2x^2 + a_4x + a_6) + (a_1x + a_3)^2 \in \vmathbb{Z}[a_1, \dots, a_6, x]$ 
hence $R = \vmathbb{Z}[a_1, \dots, a_6, x]$. Therefore, $\psi_m^2$ and $\phi_m$ are polynomials in 
$x$ over the ring $\vmathbb{Z}[a_1, \dots, a_6]$ for all $m$. For a polynomial $f$ over an arbitrary 
polynomial ring $A[x, y]$, let $\Lambda(f)$ denote the leading term of $f$ as a polynomial of $x$.
\begin{lemma}
\label{lemma:psi-lead}
$$
\Lambda(\psi_m) = 
\begin{cases}
mx^{(m^2 - 1) / 2} & \text{ if } m \text{ is odd} \\ 
\frac{m}{2}\psi_2x^{(m^2 - 4) / 2} & \text{ if } m \text{ is even}
\end{cases}
$$
\end{lemma}
\begin{proof}
We will proceed by induction on $m$. The lemma is true for $m \le 4$. Let $m = 2\ell + 1$ with 
$\ell$ even. Since $\Lambda(\psi_2^2) = 4x^3$ and $(\ell + 2)\ell^3 \ne (\ell - 1)(\ell + 1)^3$, we 
have $\Lambda(\psi_{\ell + 2}\psi_\ell^3) \ne \Lambda(\psi_{\ell - 1}\psi_{\ell + 1}^3)$. Thus
\begin{align*}
\Lambda(\psi_m) 
&= \Lambda(\psi_{\ell + 2}\psi_\ell^3 - \psi_{\ell - 1}\psi_{\ell + 1}^3) \\
&= \Lambda(\Lambda(\psi_{\ell + 2}\psi_\ell^3) - \Lambda(\psi_{\ell - 1}\psi_{\ell + 1}^3)) \\
&= \Lambda\left(\frac{\ell + 2}{2}x^{(\ell^2 + 4\ell) / 2}\frac{\ell^3}{8}x^{3(\ell^2 - 4) / 
2}(16x^6) - (\ell - 1)x^{(\ell^2 - 2\ell) / 2}(\ell + 1)^3x^{3(\ell^2 + 2\ell) / 2}\right) \\
&= ((\ell + 2)\ell^3x^{(4\ell^2 + 4\ell) / 2} - (\ell - 1)(\ell + 1)^3x^{(4\ell^2 + 4\ell) / 2}) \\
&= (2\ell + 1)x^{(4\ell^2 + 4\ell) / 2} = mx^{(m^2 - 1) / 2}
\end{align*} 
as desired. The other cases of $m$ can be verified similarly.
\end{proof}
\begin{corollary}
$\Lambda(\psi_m^2) = m^2x^{m^2 - 1}$, and $\Lambda(\phi_m) = x^{m^2}$.
\end{corollary}
\begin{proof}
The first identity is trivial from \reflemma{lemma:psi-lead}. For the second identity assume $m$ is 
odd. Then $\Lambda(x\psi_m^2) \ne \Lambda(\psi_{m - 1}\psi_{m + 1})$. Thus
\begin{align*}
\Lambda(\phi_m) 
&= \Lambda(\Lambda(x\psi_m^2) - \Lambda(\psi_{m - 1}\psi_{m + 1})) \\
&= \Lambda(m^2x^{m^2} - \frac{m - 1}{2}x^{(m^2 - 2m - 3) / 2}\frac{m + 1}{2}x^{(m^2 + 2m - 3) / 
2}(4x^3)) \\
&= m^2x^{m^2} - (m^2 - 1)x^{m^2} = x^{m^2}
\end{align*}
The case of even $m$ is treated similarly.
\end{proof}
It can be shown that polynomials $\psi_m^2$ and $\phi_m$ are coprime \cite[Sec. 1.3]{Schmitt2003}. 
Let $P = (x, y)$ be a point on the elliptic curve $E$ defined by the generalized Weierstrass 
equation, (\ref{equation:AEC}), over a field $k$ with $\fieldchar(k) \ne 2$. Then for any integer $n 
\in \vmathbb{N}$
\begin{equation}
\label{equation:mult-by-n}
[n]P = \left( \frac{\phi_n(P)}{\psi_n^2(P)}, \frac{\omega_n(P)}{\psi_n^3(P)}\right)
\end{equation}
This is usually proved by a complex analytic approach using the Weierstrass $\wp$ function, see for 
example \cite{Lang1978}. One of the important properties of the division polynomials implied from 
\refequation{equation:mult-by-n} is that if $n$ is relatively prime to $\fieldchar(k)$ then $[n]P = 
\infty$ if and only if $\psi_n(x, y) = 0$; For if $\psi_n(x, y) \ne 0$ then $\phi_n(P) / 
\psi_n^2(P), \omega_n(P) / \psi_n^3(P) \in k$ hence $[n]P \ne \infty$. Conversely, if $[n]P \ne 
\infty$ then $\psi_n(x, y) \ne 0$, since $\phi_n(P) / \psi_n^2(P) \in k$ is defined and 
$\gcd(\psi_n^2, \phi_n) = 1$.

For an elliptic curve $E$ defined over a field $k$, and a positive integer $n$, define the 
$n$-torsion subgroup of $E$ to be
$$
E[n] = \{ P \in E(\overline{k}) \mid [n]P = \infty\}
$$
which is the kernel of the endomorphism $[n]: E(\overline{k}) \rightarrow E(\overline{k})$. From 
\refequation{equation:mult-by-n}, the degree of $[n]$ is $n^2$, and by 
\refcorollary{corollary:mult-by-n}, $[n]$ is separable if and only if $\fieldchar(k) \nmid n$. So, 
$\#E[n] < n^2$ if  $\fieldchar(k) \mid n$, and $\#E[n] = n^2$ if  $\fieldchar(k) \nmid n$ by 
\refproposition{proposition:iso-ker}. For the field $k$ of characteristic $p > 0$, $E$ is called 
\textbf{ordinary} if $E[p] \cong \vmathbb{Z}_p$, and \textbf{supersingular} if $E[p] \cong 0$. The 
structure of $E[n]$ for an arbitrary $n$ is determined by the following.
\begin{theorem}
\label{theorem:n-torsion-struct}
Let $E$ be an elliptic curve over a field $K$ with $\fieldchar(K) = p$, and let $n$ be a positive 
integer. Let $m = 1$ if $p = 0$. Otherwise, let $m = p^r$ where $r$ is the largest integer such that 
$p^r \mid n$. Then
$$
E[n] = \vmathbb{Z}_m^{\delta} \oplus \vmathbb{Z}_{n / m} \oplus \vmathbb{Z}_{n / m}
$$
where $\delta = 0$ if the curve is supersingular, and $\delta = 1$ if it is ordinary.
\end{theorem}
\begin{proof}
Let $t$ be a prime divisor of $n$. Then we consider two cases for $t$: \\
Case 1: $t = p$. We have $\#E[p] < p^2$ hence $E[p] \cong 0 \text{ or } \vmathbb{Z}_p$. If $E[p] 
\cong 0$ then $E[p^k] \cong 0$ for all $k$. So let $E[p] \cong \vmathbb{Z}_p$. Since the endomorphism 
$[p]$ is surjective, there are points of order $p^j$ for all $j$. Therefore $E[p^k]$ is cyclic of 
order $p^k$ hence $E[p^k] \cong \vmathbb{Z}_{p^k}$. \\
Case 2: $t \ne p$. We have $\#E[t^k] = t^{2k}$ so that $E[t^k]$ is a finite abelian $t$-group. Every 
finite abelian $t$-group can be expressed as a direct product of cyclic groups, hence $E[t^k] \cong 
\bigoplus_{i = 1}^\ell \vmathbb{Z}_{t^{\beta_i}}$ where $\beta_i \ge 1$. Thus, $E[t^k]$ contains 
$t^\ell - 1$ elements of order $t$ which implies that $E[t] \subseteq E[t^k]$ is of order $t^\ell$ 
hence $\ell = 2$. Therefore, $E[t^k] \cong \vmathbb{Z}_{t^{\beta_1}} \oplus \vmathbb{Z}_{t^{\beta_2}}$ 
with $\beta_1 + \beta_2 = 2k$. Since $E[t^k]$ is not cyclic, we have $\beta_i \le k$ and hence 
$\beta_i = k$, $i = 1, 2$. \\
Now, assume first that $p \ne 0$. Let $n = p^\alpha p_1^{\alpha_1} \cdots p_k^{\alpha_k}$ be the 
prime factorization of $n$ in which $\alpha \ge 0$ and $\alpha_i > 0$ for $i = 1, \dots, k$, and let 
$m = p^\alpha$. Then by case 1 and 2
\begin{align*}
E[n] 
&= E[p^\alpha] \oplus \left( \bigoplus_{i = 1}^k E[p_i^{\alpha_i}] \right) = E[p^\alpha] \oplus 
\left( \bigoplus_{i = 1}^k \vmathbb{Z}_{p_i^{\alpha^i}} \oplus \vmathbb{Z}_{p_i^{\alpha^i}} \right) \\
&= E[p^\alpha] \oplus \left( \bigoplus_{i = 1}^k  \vmathbb{Z}_{p_i^{\alpha^i}} \right) \oplus \left( 
\bigoplus_{i = 1}^k  \vmathbb{Z}_{p_i^{\alpha^i}} \right) = E[p^\alpha] \oplus \vmathbb{Z}_{n / m} 
\oplus \vmathbb{Z}_{n / m} \\
&= \vmathbb{Z}_m^{\delta} \oplus \vmathbb{Z}_{n / m} \oplus \vmathbb{Z}_{n / m}
\end{align*}
If $p = 0$ then assume $n = p_1^{\alpha_1} \cdots p_k^{\alpha_k}$ is the prime factorization of $n$, 
and let $m = 1$. Then the same process results in $E[n] = \vmathbb{Z}_{n} \oplus \vmathbb{Z}_{n} = 
\vmathbb{Z}_{n / m} \oplus \vmathbb{Z}_{n / m}$ which completes the proof.
\end{proof}