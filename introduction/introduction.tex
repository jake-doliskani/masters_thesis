
\chapter{Introduction}
\label{chapter:intro}

\lettrine[findent=-0.5em, loversize=0.1]{A}{} one way function is, roughly speaking, a function that 
is easy to compute and hard to invert. The theory of one way functions forms the foundations of the 
modern cryptography \cite{Levin2003, Katz2008}. Given a certain finite cyclic group $G$, and a 
generator $g$ of $G$, the function
$$
\setlength\arraycolsep{2pt}
\begin{array}{llll}
\varphi: & \vmathbb{N}_{\le \abs{G}} & \longrightarrow & G \\
 & x & \longmapsto & g^x
\end{array}
$$
is expected to be a one way function. The problem of computing $\varphi^{-1}$ is called the 
\emph{discrete logarithm problem} (DLP) to the base $g$ in $G$. The security of many cryptographic 
schemes is based on DLP \cite{KoblitzMen2004, Odlyzko2000}. The group $G$ is suitable for this 
purpose if the DLP is hard in $G$, the elements of $G$ are easily and efficiently representable, 
computation in $G$ is fast, and the order of $G$ can be computed efficiently. As a general 
assumption, it is always assumed that $G$ has a non-smooth order, i.e. its order is a prime or a 
product of a large prime and some other very small factors; otherwise it is vulnerable to the 
Pohlig-Hellman attack \cite{Pohlig1978}. 

The traditional candidates for the group $G$ are multiplicative subgroups of finite fields. The 
arithmetic on finite fields is fast, but on the other hand, there are subexponential algorithms for 
computing discrete logarithm in these fields \cite{MenOorVan96}. Another candidate for the group 
$G$, proposed by Koblitz \cite{Koblitz1987} and Miller \cite{Miller1985} independently in 1985, is a 
cyclic subgroup of the group of points of an elliptic curve over a finite field. The advantage of 
systems based on such groups is that there is no known subexponential algorithm for DLP in these 
groups, for example the index-calculus algorithm does not apply to elliptic curves except to those 
of very special structures \cite{SilvermanSuz1998, Miller1985}. This leads to equally secure systems 
with smaller parameter sizes. The primary disadvantage of elliptic curve systems is that the 
addition of points is a relatively costly operation which is now reasonably fast due to the advances 
made in both theory and implementations, e.g. see \cite{Hankerson2004} and the references therein. 
Furthermore, Koblitz \cite{Koblitz1991, KoblitzDS1998} proposed a special family of curves allowing 
fast arithmetic. Computing the order of the group of a given elliptic curve is called point 
counting. There are polynomial time algorithms for counting points on elliptic curves, e.g. 
\cite{Schoof1985}. 

% references for Koblitz curves

As another candidate for group $G$ suitable for cryptosystems, Koblitz \cite{Koblitz1989} proposed 
the jacobian variety of a hyperelliptic curve over a finite field, which is indeed a generalization 
of the elliptic case. This leads to even smaller field sizes, and larger number of choices. Also, 
the elements of the jacobian can be represented using pairs of finite degree polynomials called the 
Mumford representation. According to Riemman-Roch theorem, there is a unique integer attached to 
every curve, called the genus of the curve. For example, a curve is elliptic if and only if it is 
genus one with at least one rational point. The main drawback of the hyperelliptic curve systems is 
that the addition on the jacobian is generally slower than the group operation on elliptic curves, 
and the higher the genus goes the slower would be the arithmetic. Moreover, for large enough genera, 
there is a version of index-calculus method for computing the discrete logarithm in the jacobian 
\cite{Adleman1994, Enge2002, Muller1999, GaudryDL2000}. 

In terms of efficiency, genus 2 curves provide the closest hyperelliptic alternative for elliptic 
curve cryptosystems \cite{Pelzl2003, Gaudry2007, TLange2005}. Furthermore, for an equally secure 
system, and the same parameter lengths, the base field for the genus 2 curve is almost twice 
smaller. To find secure hyperelliptic curves, there are two main approaches:
\begin{description}
\item[Point counting] In which one tries several random curves over a given finite field until a 
good one is found, and then computes the cardinality of the jacobian.
\item[CM method] In which, instead of trying random curves, one starts with the endomorphism ring 
and vary the base field until a curve with a good jacobian order is found, e.g. \cite[ch. 
18]{Cohen2006}, and \cite{Weng2003}. 
\end{description}
There are other methods, e.g. \cite{Sutherland2009}, that work under special conditions. There are 
also known special families of curves with good jacobian order, e.g. \cite{Koblitz1988a}. The 
complex multiplication method is efficient, but curves found by this method have more special 
structures, and it is not generally known if they are stronger or weaker than general curves. In 
this thesis, we investigate point counting on genus 2 curves.

Various approaches have been proposed for point counting on hyperelliptic curves, from which we can 
name (\textbf{\romnum{1}}) Schoof's algorithm which is the generalization of the Schoof's algorithm 
for elliptic curves, and it is polynomial time. (\textbf{\romnum{2}}) Kedlaya's algorithm 
\cite{Kedlaya2001} which is exponential time in general, but efficient in small characteristics. 
(\textbf{\romnum{3}}) Satoh's algorithm \cite{Satoh2000} which is polynomial time in small 
characteristic, but exponential in general. The best of the above approaches, for large 
characteristics, is the Schoof's algorithm. The generalization is essentially due to Pila 
\cite{Pila1990}. The Schoof's algorithm for hyperelliptic curves of genus 2 was then presented in 
Gaudry and Harley \cite{GauHar2000}, and Gaudry and Schost \cite{GauSch2004, GASC2010}. In 
particular, the following is the timeline for point counting on genus $2$ curves.
\begin{itemize}
\item Gaudry and Harley (2000): 126 bit Jacobian
\item Gaudry and Schost (2004): 164 bit, secure Jacobian
\item Sutherland (2007): 188 bit, secure Jacobian (works for special curves)
\item Gaudry and Schost (2008): 256 bit, secure Jacobian
\item Gaudry and Schost (present): 256 bit, doubly-secure Jacobian
\end{itemize}
In this thesis, we show how to improve the work of Gaudry and Schost by improving the arithmetical 
ingredients of their algorithm both theoretically and in terms of implementation. According to their 
most recent work, their algorithm makes extensive use of square (or higher) root computing, modular 
polynomial composition, and power projection. Both modular polynomial composition, and power 
projection are based on  matrix multiplication. For example, to have a fast modular polynomial 
composition, the natural choice is the algorithm of Brent and Kung \cite{BreKun1978}, which is still 
practically the best. So, we have  proposed and implemented an algorithm for fast multiplication of 
matrices with integer entries of arbitrary length. This small parallel low level library, with high 
level interfaces, is embedded into the NTL library \cite{NTL2009}. Consequently, the new modular 
composition, and power projection implementations have been embedded into NTL. For the square root 
computation, we have proposed a new algorithm, implemented it, and integrated it into NTL. This 
algorithm uses polynomial composition for computing a trace-like map that reduces the problem to the 
same problem over the prime field.

In this thesis, rather than an abstract general description of materials, specially the theory of 
elliptic and hyperelliptic curves, we have tried to conduct a concrete and intuitive approach. We 
have made an effort to make this document as self contained as possible with respect to the needed 
materials by giving clean proofs, unless the proofs are too long or there is no point in quoting 
other's proofs. The reader is assumed to have a basic knowledge of algebra, and polynomial 
arithmetic. We shall denote the asymptotic complexity bounds of polynomial multiplication, and 
modular polynomial composition by $O(M(n))$, and $O(C(n))$ respectively. The best known $M(n)$ is 
$n\log n\log\log n$ achieved by using Fast Fourier Transform (FFT) \cite{Schonhage1971}, and the 
best know $C(n)$ is $n^{1.69}$. See \refsection{section:MMapll} for more details on $C(n)$.

A brief summary of the thesis follows. 

To make the introduction of hyperelliptic curves, specially the Schoof's algorithm, smooth, and for 
the sake of completeness, Chapters \ref{chapter:ellibase} and \ref{chapter:elli-pointcount} are 
dedicated to elliptic curves. A short introduction to the theory of elliptic curves over general 
fields, including the group law, endomorphisms, division polynomials, and torsion subgroups is given 
in \refchapter{chapter:ellibase}. \refchapter{chapter:elli-pointcount} includes more specific 
properties of elliptic curves over finite fields, and it is concluded with some point counting 
methods on elliptic curves. In \refchapter{chapter:hyperelliptic}, we give an introduction to the 
theory hyperelliptic curves of general genera. \refchapter{chapter:fastMM} consists of the 
description and implementation of a fast integer matrix multiplication algorithm, and it is 
concluded with applications of this algorithm to modular polynomial composition, and power 
projection. In \refchapter{chapter:rootcomp}, we first present various algorithms, including a new 
one, for computing square roots in finite field, and then extend each of those algorithms to compute 
$k$-th roots for arbitrary $k \in \vmathbb{N}$. The last chapter consists of a quick review of a 
genus 2 Schoof algorithm, and some experimental results.