\chapter{Hyperelliptic Curves}
\label{chapter:hyperelliptic}

By the increasing applications of hyperelliptic curves in various areas of computational computer 
science, the theory of these curves has been advanced significantly during recent years. They have 
mainly been used in areas such as public-key cryptography \cite{Koblitz1989, Cohen2006}, primality 
testing \cite{AdlemanHuang1992}, integer factorization \cite{Lenstra1993, Lenstra2002}, and 
error-correcting codes \cite{LeBrigand1991}. In this chapter, we give an introduction to the basic 
theory of hyperelliptic curves. We will discuss rational functions on the curve by following a bit 
of more general theory to give a clear description of concepts like uniformizers. Then, we will give 
a brief treatment on divisors, and their basic properties. The Mumford representation, and how to 
add divisors in the set of divisor classes of degree zero, which is a group called the jacobian of 
the curve, will be discussed next. At the last section, we summarize some basic facts about 
hyperelliptic curves over finite fields.









\section{Basic definitions}

Let $k$ be field and $\overline{k}$ be its algebraic closure, and let $g \ge 2$ be an integer. A 
hyperelliptic curve of genus $g$ over $k$ is a nonsingular plane curve $\mathcal{H} \subset 
\vmathbb{A}_{k}^2$ of the form
\begin{equation}
\label{equation:hyper-general}
\mathcal{H}: y^2 + h(x)y = f(x)
\end{equation}
together with a point at infinity, where $f(x) \in k[x]$ is monic with $\deg f = 2g + 1$, and $h(x) 
\in k[x]$ with $\deg h \le g$. Here, nonsingularity means there is no point $P = (x, y) \in 
\vmathbb{A}_{\overline{k}}^2$ such that $y^2 + h(x)y - f(x) = 2y + h(x) = h'(x)y - f'(x) = 0$. When 
$g = 1$, \refequation{equation:hyper-general} is the generalized Weierstrass equation of an elliptic 
curve, see \refsection{section:W-grouplaw}. Therefore, hyperelliptic curves can be thought of as a 
generalization of elliptic curves. If $\fieldchar(k) > 2$ then the change of variables $y \mapsto y 
- \frac{1}{2}h(x)$ in (\ref{equation:hyper-general}) gives
\begin{equation}
\label{equation:hyper-not2}
\mathcal{H}: y^2 = f(x)
\end{equation}
for some monic polynomial $f(x)$ of degree $2g + 1$. This implies that $\mathcal{H}$ is nonsingular 
if there is no point $P = (x, y) \in \vmathbb{A}_{\overline{k}}^2$ satisfying $y^2 - f(x) = 2y = 
f'(x) = 0$ which is simply equivalent to saying that $f(x)$ has no repeated roots. Throughout this 
chapter, we assume $\fieldchar(k) > 2$, unless otherwise specified. For an extension $L / k$, as in 
the case of elliptic curves, the set of points in $\vmathbb{A}_L^2$ satisfying 
(\ref{equation:hyper-not2}) is denoted by $\mathcal{H}(L)$.
\begin{definition}
Let $P = (x, y)$ be a finite point on $\mathcal{H}$. The \textbf{involution} of $P$ is defined to be 
$\tilde{P} = (x, -y)$. We also define $\tilde{\infty} = \infty$. The point $P$  is special if $y = 
0$, it is called ordinary otherwise.
\end{definition}
Therefore, there are only a finite number of special points on $\mathcal{H}$, namely the points with 
roots of $f(x)$ as their first coordinate. Throughout this chapter, except for 
\refsection{section:hyper-ff}, we assume that $k = \overline{k}$, i.e. $k$ is algebraically closed.








\section{Rational functions}
\label{section:rational-functions}

Let $C \subset \vmathbb{A}_k^2$ be an affine plane curve, defined over the field $k$, and given by 
the equation $F = 0$ where $F \in k[x, y]$ is irreducible. Therefore, the ideal $\langle F \rangle 
\subset k[x, y]$ is a prime ideal. The coordinate ring of $C$, denoted by $\Gamma_k(C)$, is defined 
to be the quotient ring 
$$
\Gamma_k(C) = k[x, y] / \langle F \rangle
$$
which is an integral domain. Elements of $\Gamma_k(C)$ are called polynomial functions on $C$. We 
simply write $\Gamma(C)$ when $k$ is clear from the context. Since $\Gamma(C)$ is an integral 
domain, we can form its field of fractions denoted by $k(C)$. Elements of $k(C)$ are called 
\textbf{rational functions} on $C$, and $k(C)$ itself is called the field of rational functions on 
$C$. Every element $f \in k(C)$ is of the form $g / h$ where $g$ and $h$ are polynomial functions on 
$C$. For a rational function $f \in k(C)$, and a point $P \in C$, we say that $f$ is defined at $P$ 
if there are some polynomial functions $g, h \in \Gamma(C)$ such that $f = g / h$, and $h(P) \ne 0$. 
Otherwise, $f$ is said to have a \textit{pole} at $P$, or $P$ is called a pole of $f$, and we write 
$f(P) = \infty$.
\begin{definition}
A ring $R$ is called a local ring if it has a unique maximal ideal.
\end{definition}
Assume the set of non-units of a ring $R$ form an ideal $\mathfrak{m}$. Then $R$ is clearly a local 
ring with $\mathfrak{m}$ as its maximal ideal. This is sometime used as a definition of local rings. 
For a point $P \in C$, let $\mathcal{O}_P(C)$ denote the set of elements of $k(C)$ defined a $P$. 
Then $\mathcal{O}_P(C)$ is clearly a ring such that $\Gamma(C) \subset \mathcal{O}_P(C) \subset 
k(C)$. Let $f \in \mathcal{O}_P(C)$, and write $f = g / h$ with $g, h \in \Gamma(C)$, $h(P) \ne 0$. 
The value of $f$ at $P$ is defined as $f(P) = g(P) / h(P)$ which is independent of the choice of $g$ 
and $h$.
\begin{lemma}
$\mathcal{O}_P(C)$ is a local ring.
\end{lemma}
\begin{proof}
An element $f \in \mathcal{O}_P(C)$ is a non-unit if and only if $f(P) = 0$. Let $\mathfrak{m} = \{ 
f \in \mathcal{O}_P(C) \mid f(P) = 0\}$. Let $\phi: \mathcal{O}_P(C) \rightarrow k$ be the map $f 
\mapsto f(P)$. Then $\phi$ is surjective, and $\ker \phi = \mathfrak{m}$. Therefore, $\mathfrak{m}$ 
is a maximal ideal hence $\mathcal{O}_P(C)$ is a local ring. 
\end{proof}
A partial order on a set $\mathcal{A}$ is a reflexive, antisymmetric, and transitive relation on 
$\mathcal{A}$. Let $\mathcal{A}$ be a set partially ordered by a relation $\le$. Then the following 
statements are equivalent: i) Every non-decreasing sequence in $\mathcal{A}$ is stationary. ii) 
Every non-empty subset of $\mathcal{A}$ has a maximal element. Assume that i) is true. If ii) is 
false, then there is a non-empty subset $S$ of $\mathcal{A}$ with no maximal element. So, one can 
construct an infinite strictly increasing sequence in $T$, a contradiction. Conversely, for any 
non-decreasing sequence $x_1 \le x_2 \le \cdots$ let $S = \{ x_i, i \in \vmathbb{N} \}$. Then $S$ has 
a maximal element $x_n$ for some integer $n$, hence $x_1 \le x_2 \le \cdots \le x_n = x_{n + 1} = 
\cdots$. Now, for a ring $R$, let $\mathscr{I}$ be the set of ideals of $R$ ordered by inclusion 
$\subseteq$. Then the condition i) on $\mathscr{I}$ is called the \textit{ascending chain condition} 
(acc). The ring $R$ is said to be Noetherian if it satisfies either of i) or ii).
\begin{proposition}
A ring $R$ is Noetherian if and only if every ideal of $R$ is finitely generated.
\end{proposition}
\begin{proof}
'$\Rightarrow$'. Let $I$ be an ideal of $R$, and let $\mathscr{I}$ be the set of all finitely 
generated ideals contained in $I$. Then, by Zorn's lemma, $\mathscr{I}$ has a maximal element, say 
$J$. If $J \ne I$ then let $a \in I$, and $a \notin J$. Then $J$ is a proper ideal of the finitely 
generated ideal $J + Ra$ which is a contradiction. \\
'$\Leftarrow$'. Let $I_1 \subseteq I_2 \subseteq \cdots$ be an ascending chain of ideals of $R$. 
Then $J = \bigcup_{n = 1}^\infty I_n$ is an ideal of $R$ hence finitely generated, say by $\{ a_1, 
a_2, \dots, a_r \}$. We have $a_i \in I_{n_i}$ for some $n_i$, $i = 1 , \dots, r$. Let $k = \max_{i 
= 1}^r n_i$, then $J = I_k$, and hence we have $I_1 \subseteq I_2 \subseteq \cdots \subseteq I_k = 
I_{k + 1} = \cdots$.
\end{proof}
\begin{proposition}
$\mathcal{O}_P(C)$ is a Noetherian ring.
\end{proposition}
\begin{proof}
It can easily be seen that $\Gamma(C)$ is a Noetherian ring. Let $I$ be an ideal of 
$\mathcal{O}_P(C)$, and let $\{ a_1, a_2, \dots, a_k \}$ be a set of generators for the ideal $I 
\cap \Gamma(C)$ of $\Gamma(C)$. Now, let $a \in I$, then there is an element $c \in \Gamma(C)$ with 
$c(P) \ne 0$ such that $ca \in \Gamma(C)$. So, $ca \in I \cap \Gamma(C)$ hence $ca = \sum_{n = 1}^k 
d_na_n$ where $d_n \in \Gamma(C)$. Thus, $a = \sum_{n = 1}^k \frac{d_n}{c}a_n$ which means that 
$a_i$ generate $I$ in $\mathcal{O}_P(C)$. This completes the proof.
\end{proof}
\begin{remark}
If a commutative ring $R$ is Noetherian, and $\mathfrak{p}$ is a prime ideal of $R$, then the 
localization of $R$ with respect to $\mathfrak{p}$, denoted by $R_{\mathfrak{p}}$, is also 
Noetherian. The ring $\mathcal{O}_P(C)$ is indeed the localization of the ring $\Gamma(C)$ with 
respect to the prime ideal $\mathfrak{p} = \{ f \in \Gamma(C) \mid f(P) = 0 \}$.
\end{remark}
\begin{proposition}
\label{proposition:dvr}
Let $R$ be a Noetherian local domain that is not a field, and let $\mathfrak{m}$ be its maximal 
ideal. If $\mathfrak{m}$ is principal then there is an element $\alpha \in R$ such that every 
nonzero element $a \in R$ can be uniquely expressed in the form $a = \beta \alpha^n$ for some unit 
$\beta \in R$ and nonnegative integer $n$.
\end{proposition}
\begin{proof}
Let $\mathfrak{m} = (\alpha)$ for some non-unit $\alpha \in R$. We may assume that $a$ is not a 
unit. Then $(a) \subseteq \mathfrak{m}$. We have $a = b_1\alpha$ for some $b_1 \in R$. If $b_1$ is a 
unit we are done; otherwise let $b_1 = b_2\alpha$. The same thing holds for $b_2$ and so on. If this 
process does not terminate then we have chain of ideals $(b_1) \subseteq (b_2) \subseteq \cdots$. 
Since $R$ is Noetherian, this chain is stationary so that $(b_n) = (b_{n + 1}) = \cdots$ for some 
$n$. So, $b_{n + 1} = cb_n = c\alpha b_{n + 1}$ hence $c\alpha = 1$, and $\alpha$ is a unit which is 
a contradiction. Therefore, $a = \beta\alpha^k$ for some unit $\beta$ and some integer $k \ge 1$. 
For the uniqueness let  $\beta_1\alpha^m = a = \beta_2\alpha^n$ with $m \ge n$, and $\beta_1, 
\beta_2$ units. Then $\beta_1\alpha^{m - n} = \beta_2$ which implies $m = n$, and hence $\beta_1 = 
\beta_2$.
\end{proof}
\begin{remark}
For an ideal $\mathfrak{m} \ne (1)$ of a Noetherian domain $R$ we have $\bigcap_{n = 1}^\infty 
\mathfrak{m}^n = 0$. This means that there is an integer $k \ge 1$ such that $(a) \subseteq 
\mathfrak{m}^k$ and $(a) \nsubseteq \mathfrak{m}^{k + 1}$. Then $a = \beta \alpha^k$ for some unique 
$\beta \in R$, and $\beta$ should be a unit because $(a) \nsubseteq \mathfrak{m}^{k + 1}$. 
\end{remark}
A ring $R$ satisfying the conditions of \refproposition{proposition:dvr} is called a 
\textit{discrete valuation ring} (DVR), and the element $\alpha$ is called a \textit{uniformizer} 
for $R$. For any element $a \in R$ write $a = \beta\alpha^d$ for some unit $\beta \in R$ and some 
integer $d \ge 0$. Then the order of $a$, denoted by $\order(a)$, is defined to be $d$. There maybe 
more than one uniformizer for a discrete valuation ring $R$. Let $\alpha$ and $\lambda$ be two 
uniformizers for $R$. Then $\lambda = \beta_1\alpha^{n_1}$  , and $\alpha = \beta_2\lambda^{n_2}$. 
Thus, $\alpha = \beta_1^{n_2}\beta_2\alpha^{n_1n_2}$ which implies that $\alpha^{n_1n_2 - 1}$ is a 
unit. So, $n_1n_2 = 1$ hence $n_1 = n_2 = 1$. Therefore, the ring $R$ has a unique, up to a unit, 
uniformizer. In particular, the order of an element $a$ is independent of the choice of the 
uniformizer. 
\begin{lemma}
\label{lemma:dvr-ord}
Let $R$ be a DVR, and let $a, b \in R$. Then
\begin{enumerate}
\item 
\label{item:dvr-ord-prod}
$\order(ab) = \order(a) + \order(b)$.
\item 
\label{item:dvr-ord-sum}
$\order(a + b) \ge \min(\order(a) + \order(b))$.
\end{enumerate}
\end{lemma}
\begin{proof}
Let $\alpha$ be a uniformizer for $R$, and let $a = \beta_1\alpha^{n_1}$, and $b = 
\beta_2\alpha^{n_2}$ with $\beta_1, \beta_2 \in R$ units. Also let $n_1 \ge n_2$. Then $ab = 
\beta_1\beta_2\alpha^{n_1 + n_2}$ which proves \ref{item:dvr-ord-prod}. Also $a + b = 
\alpha^{n_2}(\beta_1\alpha^{n_1 - n_2} + \beta_2)$. We can write $(\beta_1\alpha^{n_1 - n_2} + 
\beta_2) = \beta_3\alpha^{n_3}$ for some $n_3 \ge 0$, and some unit $\beta_3$. So, $a + b = 
\beta_3\alpha^{n_1 + n_3}$ which proves \ref{item:dvr-ord-sum}.
\end{proof}
Let $K$ be the field of fractions of $R$. The definition domain of the order function on $R$ can be 
extended to $K$ in a natural way as follows. Let $f \in K$, and let $f = a / b$ for some $a, b \in 
R$. Then $\order(f) = \order(a) - \order(b)$. This is clearly independent of choices of $a$ and $b$, 
and also \reflemma{lemma:dvr-ord} remains true for elements of $K$.
\begin{theorem}
\label{theorem:un-par}
Let $P$ be a nonsingular point on $C$. Then $\mathcal{O}_P(C)$ is a discrete valuation ring. 
Moreover, any line $L$ intersecting $C$ nontangentially at $P$ is a uniformizer for 
$\mathcal{O}_P(C)$.
\end{theorem}
\begin{proof}
Let $P = (a, b)$. It can easily be seen that $\mathfrak{m} = \langle x - a, y - b\rangle$ is the 
maximal ideal of $\mathcal{O}_P(C)$. We can write 
$$
F = (x - a)\frac{\partial F}{\partial x}(P) + (y - b)\frac{\partial F}{\partial y}(P) + g(x, y)
$$
for some $g \in k[x, y]$. Since $F$ is nonsingular at $P$, either $\frac{\partial F}{\partial x}(P) 
\ne 0$ or $\frac{\partial F}{\partial y}(P) \ne 0$. Assume $\frac{\partial F}{\partial x}(P) \ne 0$. 
Then, grouping together the terms with $(x - a)$ we have $F = (x - a)r(x, y) - (y - b)^\ell s(x, y)$ 
where $r(P) \ne 0$, and $s(P) \ne 0$, and $\ell \ge 1$ is the largest integer such that 
$\frac{\partial^i F}{\partial y^i}(P) = 0$ for all $1 \le i \le \ell - 1$. Let $\bar{r}, \bar{s}$ be 
the images of $r$ and $s$ in $\mathcal{O}_P(C)$ respectively. Then $(x - a)\bar{r}(x, y) = (y - 
b)^\ell\bar{s}(x, y)$ hence $(x - a) = (y - b)^\ell\bar{s}(x, y)\bar{r}^{-1}(x, y) \in \langle y - b 
\rangle$. Thus, $\mathfrak{m} = \langle y - b \rangle$ is principal, and so, by 
\refproposition{proposition:dvr}, $\mathcal{O}_P(C)$ is a discrete valuation ring. For the second 
part of the theorem, let $L'$ be the tangent to $F$ at $P$. Since the line $L$ is distinct from 
$L'$, there is always an affine transformation taking $L, L', P$ to $y - b, x - a, P$ respectively. 
By the first part, $y - b$ is a uniformizer for $\mathcal{O}_P(C)$, hence also $L$.  
\end{proof}
Now, let $\mathcal{H}$ be the hyperelliptic curve (\ref{equation:hyper-not2}). The polynomial $F(x, 
y) = y^2 - f(x)$ is irreducible over $\overline{k}$; For the only nontrivial factorization of $F$ is 
of the form $F(x, y) = (y - a(x))(y - b(x))$ which implies that $a(x) + b(x) = 0$ hence $\deg a = 
\deg b$. Thus, $2\deg a = \deg a + \deg b = \deg f = 2g + 1$ which is a contradiction. We denote by 
$\order_{P}$  the order function on $k(\mathcal{H})$ defined by the discrete valuation ring 
$\mathcal{O}_P(\mathcal{H})$. For an ordinary point $P = (a, b)$ on $\mathcal{H}$, $\frac{\partial 
F}{\partial y}(P) = 2b \ne 0$. So, $L: x - a$ is not a tangent to $\mathcal{H}$ at $P$, and hence it 
is a uniformizer for $\mathcal{O}_P(\mathcal{H})$ by \reftheorem{theorem:un-par}. If $P = (a, 0)$ is 
a special point then $\frac{\partial F}{\partial y}(P) = 2b = 0$ hence $L: y$ is not a tangent, and 
so a uniformizer for $\mathcal{O}_P(\mathcal{H})$. 

To find a uniformizer at $P = \infty$ we need to use the projective equation of $\mathcal{H}$. Let 
$f(x) = x^{2g + 1} + a_{2g}x^{2g} + \cdots + a_1x+ a_0$. Then the projective equation is 
$\mathcal{H}: z^{2g - 1}y^2 = x^{2g + 1} + a_{2g}zx^{2g} + \cdots + a_1z^{2g}x+ a_0z^{2g + 1}$, and 
the point at infinity is $T = (0: 1: 0)$. Changing to the coordinates $w = z / y, v = x / y$ gives 
the affine curve
\begin{equation}
\label{equation:inf-uni}
\mathcal{G}: \quad w^{2g - 1} = v^{2g + 1} + a_{2g}wv^{2g} + \cdots + a_1w^{2g}v + a_0w^{2g + 1}
\end{equation}
with $Q = (0, 0)$ correspond to the point at infinity. Let $\alpha = 1 + a_{2g}(w / v) + \cdots + 
a_0(w / v)^{2g + 1}$. From (\ref{equation:inf-uni}) we have $(w / v)^{2g + 1} / \alpha = w^2$ which 
implies that $w / v = 0$ at $Q$ hence $\alpha \in \mathcal{O}_Q(\mathcal{G})$ is a unit. Again from 
(\ref{equation:inf-uni}) we have $w^{2g - 1} = v^{2g + 1}\alpha$. Let $u = v^g / w^{g - 1}$. It can 
easily be seen that $\mathfrak{m} = \langle w, v \rangle$ is the maximal ideal of 
$\mathcal{O}_Q(\mathcal{G})$. We have $u^2 = \alpha^{-1}w / v = 0$ at $Q$ hence $u \in 
\mathfrak{m}$. It can be readily verified that $v = \alpha^{g - 1}u^{2g - 1}$, and $w = 
\alpha^gu^{2g + 1}$. So, $u$ is a uniformizer for $\mathcal{O}_Q(\mathcal{G})$. Therefore, $u = v^g 
/ w^{g - 1} = x^g / yz^{g - 1}$ is a uniformizer for the projective local ring 
$\mathcal{O}_T(\mathcal{H})$, and hence $x^g / y$ is a uniformizer for the affine local ring 
$\mathcal{O}_\infty(\mathcal{H})$. From the above we have $\order_\infty(x) = \order_T(x / z) = 
\order_Q(v / w) = 2g - 1 - 2g - 1 = -2$, and $\order_\infty(y) = \order_T(y / z) = \order_Q(1 / w) = 
-2g - 1$. 
\begin{corollary}
Let $P = (a, 0)$ be a special point on $\mathcal{H}$. Then $\order_P(x - a) = 2$. In other words, $x 
- a = y^2g(x, y)$ where $g(P) \ne 0, \infty$.
\end{corollary}
\begin{proof}
It is clear from the proof of \reftheorem{theorem:un-par}.
\end{proof}
\begin{corollary}
\label{corollary:finite-zp}
Let $f \in k(\mathcal{H})^\times$ be a rational function. Then $f$ has a finite number of zeros and 
poles, and $\sum_{P \in \mathcal{H}} \order_P(f) = 0$.
\end{corollary}
\begin{proof}
It suffices to prove the statement for polynomial functions. Let $l(x) = x - a \in k[x]$, and let $P 
\in \mathcal{H}$ be a point with $a$ as its $x$-coordinate. Then $l(x)$ has only one pole of order 
$2$ at $\infty$. If $P$ is an ordinary point then $l(x)$ has a simple zero at $P$, and a simple zero 
at $\tilde{P}$; otherwise $l(x)$ has a double zero at $P$. Consequently, any polynomial $l(x) \in 
k[x]$ has a finite number of zeros and poles such that if $\deg l(x) = n$ then $\sum_{P \in 
\mathcal{H}\backslash \{\infty\}} \order_P(f) = 2n$, and $\order_\infty(f) = -2n$. 

Let $g \in \Gamma(\mathcal{H})$ be a nonzero polynomial function. Then we can write $g = a(x) + 
yb(x)$ for some polynomials $a, b \in k[x]$. Let $\bar{g} = a(x) - yb(x)$. Since the mapping 
$\varphi: \mathcal{O}_P(\mathcal{H}) \rightarrow \mathcal{O}_{\tilde{P}}(\mathcal{H})$, $\varphi(f) 
= \bar{f}$ is an isomorphism, we have $\order_P(g) = \order_{\tilde{P}}(\bar{g})$ for all $P \in 
\mathcal{H}$, and hence $\sum_{P \in \mathcal{H}} \order_P(g) = \sum_{P \in \mathcal{H}} 
\order_{\tilde{P}}(\bar{g}) = \sum_{P \in \mathcal{H}} \order_P(\bar{g})$. But $g\bar{g} \in k[x]$, 
and by above, both $g$ and $\bar{g}$ have finite number of zeros and poles, and
\[
\sum_{P \in \mathcal{H}} \order_P(g) = \frac{1}{2}\left( \sum_{P \in \mathcal{H}} \order_P(g) + 
\sum_{P \in \mathcal{H}} \order_P(\bar{g}) \right) = \frac{1}{2}\sum_{P \in \mathcal{H}} 
\order_P(g\bar{g}) = 0. \qedhere
\]
\end{proof}










\section{Divisors}

A \textit{divisor} $D$ on a hyperelliptic curve $\mathcal{H}$ is a formal sum $D = \sum_{P \in 
\mathcal{H}} n_p P$ where $n_P \in \vmathbb{Z}$, and $n_P = 0$ for almost all $P \in \mathcal{H}$. 
Therefore, the set $\mathbf{D}$ of all divisors $D$ on $\mathcal{H}$ is a free $\vmathbb{Z}$-module, 
i.e. a free abelian group. The \textit{degree} of a divisor $D$ is defined to be $\deg(D) = \sum_{p 
\in \mathcal{H}}n_P \in \vmathbb{Z}$. For divisors $D_1$ and $D_2$, we clearly have $\deg(D_1 + D_2) 
= \deg(D_1) + \deg(D_2)$. We denote by $\mathbf{D}^0$ the set of all divisors of degree zero, which 
is clearly a subgroup of $\mathbf{D}$. 
\begin{definition}
For divisors $D_1 = \sum_{P \in \mathcal{H}} m_PP$ and $D_2 = \sum_{P \in \mathcal{H}} n_PP$ we say 
$D_1 \ge D_2$ if $m_P \ge n_P$ for all $P \in \mathcal{H}$. We also define the greatest common 
divisor of $D_1$ and $D_2$ as
$$
\gcd(D_1, D_2) = \sum_{P \in \mathcal{H}} \min(n_P, m_P)(P - \infty)
$$
\end{definition}
Let $f \in k(\mathcal{H})^\times$ be a rational function. Then define the divisor of $f$ to be 
$\divisor(f) = \sum_{P \in \mathcal{H}} \order_P(f)P$, which is well defined, and has degree zero by 
\refcorollary{corollary:finite-zp}. For example, for a finite point $P = (a, b) \in \mathcal{H}$, 
$\divisor_P(x - a) = P + \tilde{P} - 2\infty$; Because if $P$ is special then $x - a$ has a double 
zero at $P = \tilde{P}$, otherwise, it has a simple zero at $P$, and simple zero at $\tilde{P}$. For 
every $f_1, f_2 \in k(\mathcal{H})$ we have $\divisor(f_1f_2) = \divisor(f_1) + \divisor(f_2)$, and 
$\divisor(f_1 / f_2) = \divisor(f_1) - \divisor(f_2)$. These are simply inherited from the order 
function. A divisor $D \in \mathbf{D}$ is said to be a \textit{principal divisor} if there is a 
rational function $f \in k(\mathcal{H})$ such that $D = \divisor(f)$.
\begin{proposition}
\label{proposition:div-const}
Let $f_1, f_2 \in k(\mathcal{H})^\times$ be rational functions. Then
\begin{enumerate}
\item 
\label{item:div-const}
$\divisor(f_1) \ge 0 \Leftrightarrow f_1 \in k$.
\item 
\label{item:div-sim}
$\divisor(f_1) = \divisor(f_2) \Leftrightarrow f_1 = cf_2$ for some $c \in k$.
\end{enumerate}
\end{proposition}
\begin{proof}
Since $\divisor(f_1) \ge 0$, $f_1 \in \mathcal{O}_P(\mathcal{H})$ for all $P \in \mathcal{H}$. Let 
$f_1(P) = c \in k$ for some $P \in \mathcal{H}$. Then we still have $\divisor(f_1 - c) \ge 0$, but 
$\deg(\divisor(f_1 - c)) > 0$ which is a impossible by \refcorollary{corollary:finite-zp} unless 
$f_1 - c = 0$ hence $f_1 = c$. This proves part \ref{item:div-const}. For part \ref{item:div-sim}, 
we have $\divisor(f_1) = \divisor(f_2) \Leftrightarrow \divisor(f_1 / f_2) = 0 \Leftrightarrow f_1 / 
f_2 \in k$ by part \ref{item:div-const}.
\end{proof}
\begin{definition}
The set of all principal divisors is a subgroup of $\mathbf{D}^0$ denoted by $\mathbf{P}$. The 
quotient group $J(\mathcal{H}) = \mathbf{D}^0 / \mathbf{P}$ is called the \textbf{jacobian} of 
$\mathcal{H}$. An element $D \in J(\mathcal{H})$ is called a divisor class of degree zero.
\end{definition}
Let $D \in \mathbf{D}$, and define
$$
\mathcal{L}(D) = \{ f \in k(\mathcal{H}) \mid \divisor(f) + D \ge 0 \} \cup \{ 0 \}
$$
It can easily be verified that $\mathcal{L}(D)$ is a vector space over $k$. Let $\ell(D) = \dim_k 
\mathcal{L}(D)$.
\begin{lemma}
\label{lemma:L-point-add}
Let $D \in \mathbf{D}$ be a divisor, and let $P \in \mathcal{H}$. Then $\dim_k(\mathcal{L}(D + P) / 
\mathcal{L}(D)) \le 1$.
\end{lemma}
\begin{proof}
It clear that $\mathcal{L}(D) \subseteq \mathcal{L}(D + P)$. Let $\alpha$ be a uniformizer for 
$\mathcal{O}_P(\mathcal{H})$, and let $n_P$ be the coefficient of $P$ in $D$. Define the mapping 
$\phi: \mathcal{L}(D + P) \rightarrow k$ by $\phi(f) = (\alpha^{n_P + 1}f)(P)$. Since $f \in 
\mathcal{L}(D + P)$, $\order_P(f) \ge -n_P - 1$ hence $\phi$ is well-defined. It can easily be 
verified that $\phi$ is a $k$-linear map with $\ker \phi = \mathcal{L}(D)$. Therefore, there exists 
an embedding $\mathcal{L}(D + P) / \mathcal{L}(D) \rightarrow k$, and the result follows.
\end{proof}
\begin{proposition}
\label{proposition:L-dim}
Let $D, D_1, D_2 \in \mathbf{D}$ be divisors on $\mathcal{H}$. Then
\begin{enumerate}
\item
\label{item:L-1}
$\mathcal{L}(0) = k$. Also $\mathcal{L}(D) = 0$ if $\deg(D) < 0$.
\item
\label{item:L-2}
If $D_1 \le D_2$ then $\mathcal{L}(D_1) \subseteq \mathcal{L}(D_2)$, and $\dim_k(\mathcal{L}(D_2) / 
\mathcal{L}(D_1)) \le \deg(D_2 - D_1)$.
\item
\label{item:L-3}
If $D_1 \equiv D_2 \pmod {\mathbf{P}}$ then $\mathcal{L}(D_1) \cong \mathcal{L}(D_2)$.
\item
\label{item:L-4}
$\ell(D) < \infty$. Moreover, if $\deg(D) \ge 0$ then $\ell(D) \le \deg(D) + 1$.
\end{enumerate}
\end{proposition}
\begin{proof}
\ref{item:L-1}. We have $f \in \mathcal{L}(0) \Leftrightarrow \divisor(f) \ge 0 \Leftrightarrow f 
\in k$ By \refproposition{proposition:div-const}.(\ref{item:div-const}). Let $\deg(D) < 0$, and $f 
\in \mathcal{L}(D)$. Then $\divisor(f) + D \ge 0$, so $0 \le \deg(\divisor(f) + D) = 
\deg(\divisor(f)) + \deg(D) = \deg(D) < 0$ which impossible unless $f = 0$. \\
\ref{item:L-2}. If $f \in \mathcal{L}(D_1)$ then $\divisor(f) + D_2 \ge \divisor(f) + D_1 \ge 0$, so 
$f \in \mathcal{L}(D_2)$ hence $\mathcal{L}(D_1) \subseteq \mathcal{L}(D_2)$. Let $D_2 = D_1 + P_1 + 
P_2 + \cdots + P_n$ where $P_i$ are not necessarily distinct. Using induction and 
\reflemma{lemma:L-point-add} we have $\dim_k(\mathcal{L}(D_2) / \mathcal{L}(D_1)) \le n = \deg(D_2 - 
D_1)$. \\
\ref{item:L-3}. Assume $D_1 = D_2 + \divisor(f)$ for some $f \in k(\mathcal{H})^\times$. Then the 
mapping $\phi: \mathcal{L}(D_1) \rightarrow \mathcal{L}(D_2)$, $\phi(g) = fg$ is an isomorphism of 
vector spaces. \\
\ref{item:L-4}. By part \ref{item:L-1}, we my assume $\deg(D) \ge 0$. Let $P \in \mathcal{H}$ be an 
arbitrary point, and let $D_3 = D - (\deg(D) + 1)P$. Then $\mathcal{L}(D_3) = 0$ by part 
\ref{item:L-1}, and hence $\ell(D) = \dim_k(\mathcal{L}(D) / \mathcal{L}(D_3)) \le \deg(D) + 1$ by 
part \ref{item:L-2}.
\end{proof}
\begin{definition}
Let $D = \sum_{P \in \mathcal{H}} n_PP$ be a divisor. Then the \textit{support} of $D$ is defined to 
be $\supp(D) = \{ P \in \mathcal{H} \mid n_P \ne 0 \}$. A \textit{semi-reduced} divisor is a divisor 
of the form $\sum_{i = 1}^k n_i(P_i - \infty)$ such that 
\begin{enumerate}[(i)]
\item $n_i \ge 0$ for all $i$,
\item if $P_i$ is a special point then $n_i \le 1$,
\item if $P_i \in \supp(D)$ then $\tilde{P_i} \notin \supp(D)$.
\end{enumerate}
If we also have $\sum_{i = 1}^k n_i \le g$, where $g$ is the genus of $\mathcal{H}$, then $D$ is 
said to be a \textit{reduced} divisor.
\end{definition}
\begin{lemma}
\label{lemma:d-eq-semi}
For every divisor $D = \sum_{i = 1}^kn_iP_i \in \mathbf{D}^0$ there is a semi-reduced divisor $D_1 = 
\sum_{i = 1}^\ell m_i(P_i - \infty)$ such that $D_1 \equiv D \pmod {\mathbf{P}}$. Moreover, $\sum_{i 
= 1}^\ell m_i \le \sum_{P_i \in \supp(D)\backslash\infty}\abs{n_i}$.
\end{lemma}
\begin{proof}
Let $D = \sum_{P \in \mathcal{H}} n_PP$. For every finite point $P = (a, b) \in \supp(D)$, if $n_P < 
0$ then the term $n_pP$ can be eliminated by subtracting the principal divisor $\divisor((x - 
a)^{n_P}) = n_P(P + \tilde{P} - 2\infty)$ from $D$. So, we obtain a divisor $D_1 = \sum m_i(P - 
\infty) \in \mathbf{D}^0$ such that $m_i \ge 0$ for all $i$, and $D_1 \equiv D \pmod {\mathbf{P}}$. 
Now, for every finite point $P = (a, b) \in \supp(D_1)$ we do the following. If $P$ is a special 
point then we subtract $\divisor((x - a)^{(m_P - \delta) / 2})$, where $\delta \equiv m_P \pmod 2$, 
from $D_1$. Otherwise, if $\tilde{P} \in \supp(D_1)$ then we subtract $\divisor((x - a)^r)$, where 
$r = \min(m_{\tilde{P}}, m_P)$, from $D_1$. This way, we obtain a divisor $D_2 \equiv D_1 \pmod 
{\mathbf{P}}$ which is clearly a semi-reduced divisor. The second part is clear by the construction.
\end{proof}
\begin{proposition}
\label{proposition:always-semi}
For every polynomial $g(x) \in k[x]$ the divisor $\divisor(g(x) - y)$ is semi-reduced. Moreover, if 
$g(x)^2 - f(x) = \prod_{i = 1}^n(x - a_i)^{m_i}$ then $\divisor(g(x) - y) = \sum_{i = 1}^nm_i(P_i - 
\infty)$ where $P_i = (a_i, g(a_i))$.
\end{proposition}
\begin{proof}
We have 
\begin{align*}
\divisor(g(x) - y) + \divisor(g(x) + y) 
&= \divisor(g(x)^2 - f(x)) = \divisor(\prod_{i = 1}^n(x - a_i)^{m_i}) \\
&= \sum_{i = 1}^n \divisor((x - a_i)^{m_i}) = \sum_{i = 1}^n m_i\divisor(x - a_i) \\
&= \sum_{i = 1}^n m_i(P_i + \tilde{P}_i - 2\infty)
\end{align*}
where $P_i = (a_i, g(a_i))$. If $P_i$ is an ordinary point then $P_i$ is a zero of $g(x) - y$ if and 
only if $\tilde{P}_i$ is a zero of $g(x) + y$, and $P_i$ or $\tilde{P}_i$ can not be a zero of both 
$g(x) - y$ and $g(x) + y$. If $P_i$ is a special point, i.e. $P_i = (a_i, 0)$, then $g(a_i) = f(a_i) 
= 0$ so that $g(x)$ has a double zero at $P_i$. Since $y$ has a simple zero at $P_i$, $g(x) - y$ has 
a simple zero at $P_i$. Also we have $(x - a_i)^2 \nmid g(x)^2 - f(x)$, because otherwise, since $(x 
- a_i)^2 \mid g(x)^2$, we have $(x - a_i)^2 \mid f(x)$ which is a contradiction because $f(x)$ has 
no multiple roots. Thus, $m_i = 1$. Putting all this together yields $\divisor(g(x) - y) = \sum_{i = 
1}^nm_i(P_i - \infty)$, and $\divisor(g(x) + y) = \sum_{i = 1}^nm_i(\tilde{P}_i - \infty)$ where 
$m_i = 1$ if $P_i$ is a special point.
\end{proof}
\begin{theorem}[\textbf{Riemann-Roch}]
\label{theorem:Riemann-Roch}
For any algebraic curve $C$, there exists a divisor $\omega$ and an integer $g$ such that for any 
divisor $D$ on $C$,
$$
\ell(D) = \deg(D) - g + 1 + \ell(\omega - D)
$$
\end{theorem}
\begin{proof}
See \cite[ch. 1]{Lang1982} or \cite[ch. 12]{Bump1998}.
\end{proof}
\begin{remark}
The divisor $\omega$ is the divisor of a differential on $C$, and $g$ is called the genus of $C$.
\end{remark}
\begin{theorem}
\label{theorem:unique-redu}
For any divisor $D \in \mathbf{D}^0$, there exists a unique reduced divisor $D_1$ such that $D_1 
\equiv D \pmod {\mathbf{P}}$.
\end{theorem}
\begin{proof}
(Existence) By \reftheorem{theorem:Riemann-Roch}, $\ell(D) \ge \deg(D) - g + 1$ for any divisor 
$D$.\footnote{This is called Riemann inequality.} Replacing $D$ by $D + g\infty$ we have $\ell(D + 
g\infty) \ge g - g + 1 = 1$. This means that there is a rational function $f \in 
k(\mathcal{H})^\times$ such that $\divisor(f) + D + g\infty \ge 0$. Let $D_1 = \divisor(f) + D$. 
Then $D_1 + g\infty \ge 0$ which means that $D_1$ is of the form $\sum_in_i(P_i - \infty)$ with $n_i 
\ge 0$, and $\sum_in_i = g$. The result now follows from \reflemma{lemma:d-eq-semi}. \\
(Uniqueness) Letting $D = 0$ in \reftheorem{theorem:Riemann-Roch}, we have $\ell(0) = 0 - g + 1 + 
\ell(\omega)$. By \refproposition{proposition:L-dim}.(\ref{item:L-1}), $\ell(0) = 1$ hence 
$\ell(\omega) = g$. Similarly, $\ell(\omega) = \deg(\omega) - g + 2$ by setting $D = \omega$. 
Therefore, $\deg(\omega) = 2g - 2$. Now, let $D_1 \equiv D_2 \pmod {\mathbf{P}}$ be two reduced 
divisors. Assume that $D_1 \ne D_2$, and let $D = D_1 - D_2$ be a principal divisor. By 
\reflemma{lemma:d-eq-semi}, there is a principal divisor $D_3 \equiv D \pmod {\mathbf{P}}$ with $D_3 
+ 2g\infty \ge 0$. It can easily be verified that $D_3 \ne 0$ by the proof of 
\reflemma{lemma:d-eq-semi}. Let $\divisor(f) = D_3$ for some $f \in k(\mathcal{H})^\times$. Then $f 
\in \mathcal{L}(2g\infty)$. Letting $D = 2g\infty$ in \reftheorem{theorem:Riemann-Roch}, we have 
$\ell(2g\infty) = g + 1 - \ell(\omega - 2g\infty)$. Since $\deg(\omega - 2g\infty) = -2$ by above, 
$\ell(\omega - 2g\infty) = 0$ by \refproposition{proposition:L-dim}.(\ref{item:L-1}), and hence 
$\ell(2g\infty) = g + 1$. But, $x^i \in \mathcal{L}(2g\infty)$ for all $i = 0, \dots, g$. Since $1, 
x, \dots, x^g$ have poles of different order, they are linearly independent. So, they form a basis 
for $\mathcal{L}(2g\infty)$. This means that $f$ is a function in $x$. If $f$ in nonconstant then it 
has a root $a \in k$. Let $P = (a, b)$ be a point on $\mathcal{H}$. If $P$ is ordinary then 
$\tilde{P}$ is also a zero of $f$ hence $P, \tilde{P} \in \supp(D_3)$ which is a contradiction, 
since $D_3$ is semi reduced. If $P$ is special then $f$ has a double zero at $P$ hence the 
coefficient of $P$ in $D_3$ is at least $2$, contradiction again. Thus, $f$ is constant hence $D_3 = 
0$, a contradiction. 
\end{proof}
\begin{remark}
The group $J(\mathcal{H})$ can be considered as an algebraic variety. It is, indeed, an abelian 
variety of dimension $g$ called the jacobian variety. For any point $P \in \mathcal{H}$, the divisor 
$P - \infty$ is reduced. Therefore, by \reftheorem{theorem:unique-redu}, the mapping
$$
\setlength\arraycolsep{2pt}
\begin{array}{llll}
\varphi: & \mathcal{H} & \longrightarrow & J(\mathcal{H}) \\
 & P & \longmapsto & P - \infty
\end{array}
$$
is an embedding of $\mathcal{H}$ into its jacobian. In the case of elliptic curves, this is clearly 
an isomorphism, hence an elliptic curve is an abelian variety of dimension 1. If $\ell$ is an 
integer such that $\fieldchar(k) \nmid \ell$; then the $\ell$-torsion subgroup of $J(\mathcal{H})$, 
denoted by $J(\mathcal{H})[\ell]$, is isomorphic to $(\vmathbb{Z} / \ell\vmathbb{Z})^{2g}$, see 
\cite[ch. 4, 5]{Cohen2006}. When $g = 1$, i.e. $\mathcal{H}$ is an elliptic curve, this is a special 
case of \reftheorem{theorem:n-torsion-struct}. Analogous to the case of elliptic curves, there are 
\emph{division polynomials} for hyperelliptic curves closely related to the torsion elements of 
$J(\mathcal{H})$. In 1994, Cantor \cite{Cantor1994} gave these polynomials defined by efficiently 
computable recurrences.
\end{remark}










\section{Mumford Representation}

By \reftheorem{theorem:unique-redu}, the elements of the group $J(\mathcal{H})$ are indeed reduced
divisors. So, theoretically, given two reduced divisors $D_1, D_2$, we can add them, and reduce the 
result to obtain another reduced divisor. However, representing elements of $J(\mathcal{H})$ by 
divisor classes, i.e by formal sums of points on $\mathcal{H}$, is not computationally very useful. 
In this section, we present a concrete representation of the elements of $J(\mathcal{H})$ by pairs 
of polynomials, which was proposed by Mumford \cite{Mumford1984}. 

Let $P = (a, b)$ be a point on $\mathcal{H}$, and let $g \in k(\mathcal{H})$ be a rational function. 
Let $\mathcal{O}_P(\mathcal{H})$ be the local ring at $P$, and let $t$ be a uniformizer for it. Then 
$g = t^mh$ for a unique, not necessarily positive, integer $m$, and rational function $h \in 
k(\mathcal{H})$ such that $h(P) \ne 0, \infty$. If $h$ is not a constant then let $h(P) = c \in k$. 
Then $h(x, y) - c$ has a zero at $P$, so $h(x, y) - c = t^{m_1}h_1(x, y)$ for a unique integer $m_1 
\ge 1$, and a rational function $h_1$ such that $h_1(P) \ne 0, \infty$. Hence $h(x, y) = c + th_1(x, 
y)$. The same thing can be done to $h_1$, and so on. Therefore, for any given $k \ge m$, we can 
uniquely write 
\begin{equation}
\label{equation:u-p-series}
g(x, y) = \sum_{n = m}^{k - 1} c_nt(x, y)^n + t(x, y)^kh_k(x, y)
\end{equation}
where $m \in \vmathbb{Z}$, and $h_k \in k(\mathcal{H})$ with $h_k(P) \ne 0, \infty$.
\begin{proposition}
\label{proposition:mum-sq}
Let $f$ be the polynomial in \ref{equation:hyper-not2}. For any ordinary point $P = (a, b) \in 
\mathcal{H}$, and any integer $k \ge 1$, there exists a unique polynomial $g \in k[x]$ such that 
$g(a) = b$, and $g(x)^2 \equiv f(x) \mod {(x - a)^k}$.
\end{proposition}
\begin{proof}
Since $P$ is ordinary, and $y$ is a polynomial function, $m = 0$, and $t = (x - a)$ in 
(\ref{equation:u-p-series}). Thus, $y = \sum_{n = 0}^{k - 1} b_n(x - a)^n + (x - a)^kh(x, y)$ where 
$b_0 = b$. Let $g(x) = \sum_{n = 0}^{k - 1} b_n(x - a)^n$. Then $g(a) = b_0 = b$ and obviously 
$g(x)^2 \equiv y^2 \equiv f(x) \mod (x - a)^k$.
\end{proof}
\begin{theorem}
\label{theorem:mum-rep}
Let $\mathcal{S}$ be the set of all semi-reduced divisors on $\mathcal{H}$, and let $\mathcal{P}$ be 
the set of pairs of of polynomials $(u, v) \in k[x] \times k[x]$ such that \emph{(\romnum{1})} $u$ 
is monic, and $\deg v < \deg u$ \emph{(\romnum{2})} $u \mid v^2 - f$. Then the following mapping is 
a bijection.
\begin{equation}
\label{equation:Mum-corr}
\setlength\arraycolsep{2pt}
\begin{array}{llll}
\psi: & \mathcal{P} & \longrightarrow & \mathcal{S} \\
& (u, v) & \longmapsto & \gcd(\divisor(u), \divisor(v - y))
\end{array}
\end{equation}
\end{theorem}
\begin{proof}
Let $(u, v)$ be a pair of polynomials as in the theorem. By 
\refproposition{proposition:always-semi}, $\divisor(v - y)$ is semi-reduced hence $\gcd(\divisor(u), 
\divisor(v - y))$ is semi-reduced. \\
$\psi$ is \textbf{surjective}: Let $D = \sum_{i = 1}^km_i(P_i - \infty) \in \mathcal{S}$, where $P_i 
= (a_i, b_i)$, be a semi-reduced divisor. Let $u(x) = \prod_{i = 1}^k(x - a_i)^{m_i}$. We will find 
a set of polynomials $v_i(x)$, $i = 1, \dots, k$ such that $v_i(x)^2 \equiv f(x) \mod (x - 
a_i)^{m_i}$, and $v_i(a_i) = b_i$. If $P_i$ is a special point then $m_i = 1$. Set $v_i(x) = 0$. 
Then we have $v_i(x)^2 = 0 \equiv f \mod (x - a_i)$, and $v_i(a_i) = 0 = b_i$. If $P_i$ is ordinary 
then let $v_i(x)$ be the polynomial $g(x)$ in \refproposition{proposition:mum-sq} with $k = m_i$. By 
the Chinese remaindering theorem, there exists a a unique polynomial $v(x) \in k[x]$ such that $v(x) 
\equiv v_i(x) \mod (x - a_i)^{m_i}$ for all $i$, and $\deg v(x) < \sum_{i = 1}^k m_i = \deg u(x)$. 
We clearly have $u \mid v^2 - f$. Let $v(x)^2 - f(x) = \prod_{i = 1}^k(x - a_i)^{m_i}\prod_{j = 
1}^\ell (x - c_j)^{n_i}$. Then by \refproposition{proposition:always-semi}, $\divisor(v(x) - y) = D 
+ \sum_{j = 1}^\ell n_i(P_j - \infty)$ where $P_j = (c_j, v(c_j))$. Therefore, $D = 
\gcd(\divisor(u), \divisor(v - y))$ hence $\psi(u, v) = D$. \\
$\psi$ is \textbf{injective}: Let $\psi(u_1, v_1) = \psi(u_2, v_2)$. By construction $u_1 = u_2$. We 
have $v_1(x) - v_2(x) = (v_1(x) - y) - (v_2(x) - y)$. So, $v_1(x) - v_2(x)$ vanishes at least to 
order $m_i$ at $P_i$ hence it has at least $\sum_{i = 1}^k m_i = \deg u(x)$ zeros. This is a 
contradiction since $\deg(v_1(x) - v_2(x)) < \deg u(x)$. Therefore $v_1(x) - v_2(x)$ is the zero 
polynomial.
\end{proof}
A direct corollary of \reftheorem{theorem:unique-redu}, and the proof of 
\reftheorem{theorem:mum-rep} is that every reduced divisor $D$, i.e. every element of 
$J(\mathcal{H})$, corresponds to a pair $(u, v)$ as in (\ref{equation:Mum-corr}), with $\deg u \le 
g$. This pair is called the \textbf{Mumford Representation} of $D$.









\section{Addition on the jacobian $J(\mathcal{H})$}
\label{section:add-jac}

Assume the elements of $J(\mathcal{H})$, i.e the reduced divisors, are given by their Mumford 
representations. In this section, we present an algorithm due to Cantor \cite{Cantor1987} for 
addition on $J(\mathcal{H})$ based on the Mumford representation. Let us first see how to obtain the 
reduced divisor corresponding to a given semi-reduced divisor.
\begin{algorithm}
[Reduction of semi-reduced divisors]
\label{algorithm:semi-d-redu}
\begin{algorithmic}[1]
\REQUIRE A pair $(u, v)$ representing a semi-reduced divisor $D$.
\ENSURE  A pair representing the reduced divisor $D_1 \equiv D \pmod {\mathbf{P}}$.
\WHILE {$\deg(u) > g$}
	\STATE 
	\label{step:redu-div-u_1}
	$u_1 \leftarrow (v^2 - f) / u$
	\STATE
	\label{step:redu-v-redu}
	$v_1 \leftarrow -v \mod u_1$
	\STATE
	\label{step:redu-u-monic}
	make $u_1$ monic by dividing it by its leading coefficient.
	\STATE $u \leftarrow u_1$, $v \leftarrow v_1$.
\ENDWHILE
\RETURN $(u, v)$
\end{algorithmic}
\end{algorithm}

\begin{theorem}
\refalgorithm{algorithm:semi-d-redu} works correctly.
\end{theorem}
\begin{proof}
We first show that the terminates. Assume $\deg u \ge g + 1$. Then $\deg(v^2 - f) \le \max(\deg v^2, 
\deg f) < \max(\deg u^2, \deg u^2) = 2\deg u$. Thus, $\deg u_1 = \deg(v^2 - f) - \deg u < 2\deg u - 
\deg u = \deg u$. Therefore, the degree of $u$ decreases strictly by each iteration, hence the while 
loop iterates finitely many times. \\
Let $D_1 = \gcd(\divisor(u_1), \divisor(v_1 - y))$ where $u_1, v_1$ are as in steps 
\ref{step:redu-div-u_1} and \ref{step:redu-v-redu} of the algorithm repectively. Then $u_1$ is 
monic, and $\deg v_1 < \deg u_1$ by steps \ref{step:redu-v-redu} and \ref{step:redu-u-monic}. Also 
$v_1^2 - f \equiv v^2 - f \equiv 0 \pmod {u_1}$. Therefore, $D_1$ is semi-reduced by 
\reftheorem{theorem:mum-rep}. \\
For a divisor $D = \sum_{P \in \mathcal{P}}n_PP$, let $\widetilde{D} = \sum_{P \in \mathcal{P}} 
n_P\tilde{P}$. Then $\divisor(u) = D + \widetilde{D}$. By \refproposition{proposition:always-semi}, 
$\divisor(v - y) = D + K$, and $\divisor(v + y) = \widetilde{D} + \widetilde{K}$, where $K = \sum_{i 
= 1}^\ell k_i(P_i - \infty)$, for some $P_i \in \mathcal{H}$, and $k_i \ge 0$. Thus, $\divisor(u_1) 
= \divisor((v^2 - f) / u)) = \divisor(v^2 - f) - \divisor(u) = D + K + \widetilde{D} + \widetilde{K} 
- D - \widetilde{D} = K + \widetilde{K}$. Since $\divisor(v + y)$ is semi-reduced, 
$\supp\{\widetilde{D} \backslash \infty \} \cap \supp\{K \backslash \infty \} = \varnothing$, which 
implies that $\gcd(\divisor(u_1), \divisor(v + y)) = \widetilde{K}$ hence $D_1 = \gcd(\divisor(u_1), 
\divisor(v_1 + y)) = \widetilde{K}$. On the other hand, $D - D_1 = D - \widetilde{K} = \divisor(v - 
y) - K - \divisor(u_1) + K = \divisor(v - y) - \divisor(u_1)$ which means that $D_1 \equiv D \pmod 
{\mathbf{P}}$.
\end{proof}

\begin{algorithm}
[Cantor's algorithm for adding reduced divisors]
\label{algorithm:Cantor-add}
\begin{algorithmic}[1]
\REQUIRE Pairs $(u_1, v_1)$, $(u_2, v_2)$ representing reduced divisors $D_1$, and $D_2$ 
respectively.
\ENSURE  A pair $(u, v)$ representing the reduced divisor $D \equiv D_1 + D_2 \pmod {\mathbf{P}}$.
\STATE compute $d_1, r_1, r_2$ such that $d_1 = \gcd(u_1, u_2)$, and $d_1 = r_1u_1 + r_2u_2$ using 
the extended Euclidean algorithm.
\STATE compute $d, s_1, s_2$ such that $d = \gcd(d_1, v_1 + v_2)$, and $d = s_1d_1 + s_2(v_1 + v_2)$ 
using the extended Euclidean algorithm.
\STATE let $g_1 = s_1r_1$, $g_2 = s_2r_2$, and $g_3 = s_2$ so that $d = g_1u_1 + g_2u_2 + g_3(v_1 + 
v_2)$.
\STATE $u \leftarrow u_1u_2 / d^2$.
\STATE $v \leftarrow (u_1v_2g_1 + u_2v_1g_2 + (v_1v_2 + f)g_3) / d \mod u$
\STATE use \refalgorithm{algorithm:semi-d-redu} to reduce $(u, v)$
\RETURN $(u, v)$
\end{algorithmic}
\end{algorithm}

\begin{theorem}
\refalgorithm{algorithm:Cantor-add} works correctly.
\end{theorem}
\begin{proof}
The algorithm first generates a pair $(u, v)$ representing a semi-reduced divisor $D \equiv D_1 + 
D_2 \pmod {\mathbf{P}}$, and then uses \refalgorithm{algorithm:semi-d-redu} to obtain the desired 
reduced divisor. Here, we omit the lengthy proof, and refer the reader to \cite[ch. 
13]{Washington2008} or \cite[appendix]{Koblitz1998}.
\end{proof}









\section{Hyperelliptic curves over $\vmathbb{F}_q$}
\label{section:hyper-ff}

In this section we assume the hyperelliptic curve $\mathcal{H}$, \refequation{equation:hyper-not2}, 
is defined over the finite field $k = \vmathbb{F}_q$ where $q = p^n$ with $p \ge 3$ prime. Let 
$\phi_q$ be the $q$-th power Frobenius map on $\mathcal{H}$. For a divisor $\sum_{P \in \mathcal{H}} 
n_PP$, we define $\phi_q(D) = \sum_{P \in \mathcal{H}}n_P\phi_q(P)$. Also for a rational function $g 
\in \overline{\vmathbb{F}_q}(\mathcal{H})$, let $g^{\phi_q}$ be $g$ with $\phi_q$ applied to its 
coefficients. A divisor $D$ is said to be defined over $\vmathbb{F}_q$ if $\phi_q(D) = D$. 
Consequently, a divisor class $D \in J(\mathcal{H})$ is defined over $\vmathbb{F}_q$ if $\phi_q(D) 
\equiv D \pmod {\mathbf{P}}$. We denote by $J_{\vmathbb{F}_q}(\mathcal{H})$, the elements of 
$J(\mathcal{H})$ defined over $\vmathbb{F}_q$. The following is a consequence of 
\reftheorem{theorem:mum-rep} for the restriction $\vmathbb{F}_q$ of $k = \overline{\vmathbb{F}_q}$.
\begin{proposition}
Let $\mathcal{P}_{\vmathbb{F}_q}$ be the set of pairs of of polynomials $(u, v) \in \vmathbb{F}_q[x] 
\times \vmathbb{F}_q[x]$ such that \emph{(\romnum{1})} $u$ is monic, and $\deg v < \deg u \le g$ 
\emph{(\romnum{2})} $u \mid v^2 - f$. Then the following mapping is a bijection.
\begin{equation}
\label{equation:Mum-corr-fq}
\setlength\arraycolsep{2pt}
\begin{array}{llll}
\psi_{\vmathbb{F}_q}: & \mathcal{P}_{\vmathbb{F}_q} & \longrightarrow & J_{\vmathbb{F}_q}(\mathcal{H}) 
\\
& (u, v) & \longmapsto & \gcd(\divisor(u), \divisor(v - y))
\end{array}
\end{equation}
\end{proposition}
\begin{proof}
Let $D \in J_{\vmathbb{F}_q}(\mathcal{H})$, and let $E$ be the unique reduced divisor in the class of 
$D$. Then $D - E = \divisor(g)$ for some function $g$, and hence $\phi_q(D)- \phi_q(E) = 
\divisor(\phi_q(F))$, which implies that $\phi_q(E)$ is in the class of $\phi_q(D)$. By uniqueness, 
$E = \phi_q(E)$. Let $(u, v)$ be the Mumford representation of $E$. Then $(u^{\phi_q}, v^{\phi_q})$ 
is the Mumford representation of $\phi_q(E)$, hence $(u, v) = (u^{\phi_q}, v^{\phi_q})$. Therefore, 
$u, v \in \vmathbb{F}_q[x]$. Conversely, let $u, v \in \vmathbb{F}_q$, and let $D = 
\psi_{\vmathbb{F}_q}(u, v)$. Then 
\begin{align*}
\phi_q(D) 
&= \phi_q(\gcd(\divisor(u), \divisor(v - y))) = \gcd(\phi_q(\divisor(u)), \phi_q(\divisor(v - y))) 
\\
&= \gcd(\divisor(\phi_q(u)), \divisor(\phi_q(v - y))) = \gcd(\divisor(u), \divisor(v - y)) = D 
\qedhere
\end{align*}
\end{proof}
Since there are finitely many polynomials in $\vmathbb{F}_q[x]$ of degree less than or equal to $g$, 
$J_{\vmathbb{F}_q}(\mathcal{H})$ is a finite set. It is clear that $J_{\vmathbb{F}_q}(\mathcal{H})$ is 
closed under addition and inversion, hence it is a group. As in the case of elliptic curves, 
$\phi_q$ is an inseparable endomorphism of $J(\mathcal{H})$ of degree $q^g$. We have 
$J_{\vmathbb{F}_q}(\mathcal{H}) = \ker(\phi_q - 1)$, and hence $\#J_{\vmathbb{F}_q}(\mathcal{H}) = 
\#\ker(\phi_q - 1) = \deg(\phi - 1)$. Denote by $\chi_q(t)$ the characteristic polynomial of 
$\phi_q$. It can be shown that $\chi_q(t) \in \vmathbb{Z}[t]$ is monic of degree $2g$, and that 
$\deg(\phi_q - 1) = \chi_q(1)$. For an integer $n$ prime to $p$, the restriction 
$\phi_q\vert_{J(\mathcal{H})[n]}$ has the characteristic polynomial $\chi_q(t) \pmod n$; see 
\cite[ch. 4, 5]{Cohen2006} or \cite{Mumford1974}.

Let $N_k$ denote the number of $\vmathbb{F}_{q^k}$-rational points of $\mathcal{H}$. Define the 
generating function
\begin{equation}
\label{equation:zeta}
Z(\mathcal{H}, t) = \exp\left( \sum_{k = 1}^\infty \frac{N_k}{k}t^k \right)
\end{equation}
which is called the \textbf{zeta function} of $\mathcal{H}$ over $\vmathbb{F}_q$. The function 
$Z(\mathcal{H}, t)$ is of fundamental importance in the theory of hyperelliptic curves over finite 
fields. The following theorem states some known properties of this function.
\begin{theorem}[\textbf{Weil Conjectures}]
\label{theorem:Weil-conj}
Let $\mathcal{H}$ be a hyperelliptic curve of genus $g$ over $\vmathbb{F}_q$, and $Z(\mathcal{H}, t)$ 
be the zeta function of $\mathcal{H}$.
\begin{enumerate}
\item 
$Z(\mathcal{H}, t) \in \vmathbb{Q}[[t]]$ is a rational function.
\item
\label{item:zeta-func-equ}
$Z(\mathcal{H}, t)$ satisfies the functional equation 
$$
Z\left( \mathcal{H}, \frac{1}{qt} \right) = q^{1 - g}t^{2 - 2g}Z(\mathcal{H}, t)
$$
\item
\label{item:zeta-frac}
We have
\begin{equation}
\label{equation:zeta2}
Z(\mathcal{H}, t) = \frac{L(t)}{(1 - t)(1 - qt)}
\end{equation}
where $L(t) \in \vmathbb{Z}[t]$ is of degree $2g$ such that $L(t) = \prod_{i = 1}^{2g} (1 - 
\alpha_it)$ where $\alpha_i$ are algebraic integers such that $\alpha_{g + i} = \overline{\alpha}_i$ 
and $\abs{\alpha_i} = \sqrt{q}$ for $i = 1, \dots, g$.
\end{enumerate}
\end{theorem}
\begin{proof}
See \cite[ch. 5]{Stichtenoth2009} or \cite[part \Romnum{2}]{vanderGeer1988}.
\end{proof}
\begin{corollary}
\label{corollary:Lfunct-form}
$L(t)$ is of the form $L(t) = a_0 + a_1t + \cdots + a_{2g}t^{2g}$ where $a_0 = 1, a_{2g} = q^g$, and 
$a_{2g - i} = q^{g - i}a_i$ for $i = 0, \dots, g$.
\end{corollary}
\begin{proof}
The functional equation of $Z(\mathcal{H}, t)$, 
\reftheorem{theorem:Weil-conj}.(\ref{item:zeta-func-equ}), gives 
$$
L(t) = q^gt^{2g}L\left( \frac{1}{qt} \right) = \frac{a_{2g}}{q^g} + \frac{a_{2g - 1}}{q^{g - 1}}t + 
\cdots + q^ga_0t^{2g}
$$
and by \reftheorem{theorem:Weil-conj}.(\ref{item:zeta-frac}), $a_0 = 1$.
\end{proof}
\begin{corollary}
\label{corollary:num-point-H}
Let $\alpha_i$, $i = 1, \dots, 2g$ be as in \reftheorem{theorem:Weil-conj}.(\ref{item:zeta-frac}). 
Then $\#\mathcal{H}(\vmathbb{F}_{q^n}) = q^n + 1 - \sum_{i = 1}^{2g} \alpha_i^n$.
\end{corollary}
\begin{proof}
Taking the logarithm of both sides of (\ref{equation:zeta2}), gives 
\begin{align*}
\ln(Z(\mathcal{H}, t)) 
&= \sum_{i = 1}^{2g} \ln(1 - \alpha_it) - \ln(1 - t) - \ln(1 - qt) \\
&= -\sum_{i = 1}^{2g} \sum_{k = 1}^\infty \frac{\alpha_i^kt^k}{k} + \sum_{k = 1}^\infty 
\frac{t^k}{k} + \sum_{k = 1}^\infty \frac{q^kt^k}{k} \\
&= \sum_{k = 1}^\infty \frac{1}{k}(q^k + 1 - \sum_{i = 1}^{2g} \alpha_i^k)t^k
\end{align*}
Comparing this to (\ref{equation:zeta}) yields the result.
\end{proof}
It can be shown that $\chi_q(t) = t^{2g}L(1 / t)$. Therefore, when $g = 1$, 
\refcorollary{corollary:num-point-H} gives \reftheorem{theorem:Weil-subfc}. By 
\refcorollary{corollary:Lfunct-form},
\begin{equation}
\label{equation:Frob-charpoly}
\chi_q(t) = t^{2g} + a_1t^{2g - 1} + \cdots + a_{g - 1}t^{g - 1} + a_gt^g + a_{g - 1}qt^{g - 1} + 
\cdots + a_2q^{g - 2}t^2 + a_1q^{q - 1}t + q^g 
\end{equation}
In particular $\chi_q(t) = \prod_{i = 1}^{2g} (t - \alpha_i)$ which means that the eigenvalues of 
$\phi_q$ are $\alpha_i$, $i = 1, \dots, 2g$. Therefore, the eigenvalues of $\phi_{q^k} = \phi_q^k$ 
are $\alpha_i^k$, $i = 1, \dots, 2g$. Hence $\chi_{q^k}(t) = \prod_{i = 1}^{2g} (t - \alpha_i^k)$. 
As mentioned earlier, $\#J_{\vmathbb{F}_q}(\mathcal{H}) = \chi_q(1)$, consequently, 
\begin{equation}
\label{equation:num-point-J}
\#J_{\vmathbb{F}_{q^k}}(\mathcal{H}) = \chi_{q^k}(1) = \prod_{i = 1}^{2g} (1 - \alpha_i^k)
\end{equation}
\begin{theorem}[\textbf{Weil bounds}]
Let $\mathcal{H}$ be a hyperelliptic curve of genus $g$ over $\vmathbb{F}_q$. Then
$$
\renewcommand{\arraystretch}{1.25}
\begin{array}{c}
\abs{\#\mathcal{H}(\vmathbb{F}_{q^n}) - (q^n + 1)} \le 2g\sqrt{q^n} \:, \\
(\sqrt{q^n} - 1)^{2g} \le \#J_{\vmathbb{F}_{q^n}}(\mathcal{H}) \le (\sqrt{q^n} + 1)^{2g}
\end{array}
$$
\end{theorem}
\begin{proof}
By \refcorollary{corollary:num-point-H} and \reftheorem{theorem:Weil-conj}.(\ref{item:zeta-frac}), 
$$
\abs{\#\mathcal{H}(\vmathbb{F}_{q^n}) - (q^n + 1)} = \abs{\sum_{i = 1}^{2g} \alpha_i^n} \le \sum_{i = 
1}^{2g} \abs{\alpha_i^n} = \sum_{i = 1}^{2g} \abs{\alpha_i}^n = 2g\sqrt{q^n}
$$
which proves the first part. For the second part, by (\ref{equation:num-point-J}) and 
\reftheorem{theorem:Weil-conj}.(\ref{item:zeta-frac}),
$$
\#J_{\vmathbb{F}_{q^k}}(\mathcal{H}) = \abs{\prod_{i = 1}^{2g} (1 - \alpha_i^k)} = \prod_{i = 1}^{2g} 
\abs{(1 - \alpha_i^k)} \le \prod_{i = 1}^{2g} (1 + \abs{\alpha_i^k}) = (\sqrt{q^n} + 1)^{2g}
$$
which establishes the upper bound. The lower bound is established the same way. 
\end{proof}